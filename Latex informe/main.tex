% /////////////////////////////////////////////////////////////////
% ///////                      PREÁMBULO                    ///////
% /////////////////////////////////////////////////////////////////


\documentclass[a4paper,openany,oneside]{book}

% \documentclass[a4paper,openany,oneside]{scrbook}


% Paquetes para el idioma y fuentes
%\usepackage[english]{babel}
\usepackage[spanish]{babel}
\usepackage[utf8x]{inputenc}
\usepackage[T1]{fontenc}
\addto\captionsspanish{\renewcommand{\tablename}{Tabla}}


% Configuración de margenes (opcional)
\usepackage[a4paper,top=2.5cm,bottom=2.5cm,left=3cm,right=3cm,marginparwidth=1.75cm]{geometry}
% \usepackage{atbegshi}
% \AtBeginShipoutNext{\thispagestyle{plain}}


%% Useful packages
\usepackage{amsmath}
\usepackage{graphicx}   % Required for inserting images
\usepackage{array}      % Para la definición de columnas con ancho específico
\usepackage[table,xcdraw]{xcolor} % Para el color de las celdas
\usepackage{adjustbox}
\usepackage{listings}
\usepackage[hidelinks]{hyperref}
\usepackage{pdfpages}
\usepackage{svg}
\usepackage{placeins}
\usepackage{pdflscape}
\usepackage{afterpage}

 %opciones para las imagenes
\usepackage{float}   
\usepackage{caption}
\captionsetup[figure]{skip=0pt} % Ajusta la distancia entre la imagen y el caption

% Ajuste de interlineado entre parrafos y sangria
\usepackage{parskip}
\usepackage{titlesec}

% \setlength{\parskip}{1em} 
% \setlength{\parindent}{30pt} 
% \titlespacing{\subsection}{0pt}{1em}{0pt}
% \titlespacing{\section}{0pt}{2em}{0pt}
% \titlespacing*{\chapter}{0pt}{-20pt}{20pt}

\usepackage[T1]{fontenc}
\usepackage{amssymb}
\usepackage{newunicodechar}
\newunicodechar{⟪}{\llangle}
\newunicodechar{⟫}{\rrangle}





\begin{document}

% /////////////////////////////////////////////////////////////////
% ///////                      CARÁTULA                     ///////
% /////////////////////////////////////////////////////////////////

\begin{titlepage}
    \centering
    \vspace*{3cm}
    
    {\Huge\bfseries Entrega Final\\[0.4cm] Calidad de Datos e Información\par}
    
    \vspace{1.5cm}
    
    {\LARGE\bfseries Context-aware Data Quality\\Methodology (CaDQM)\par}
    
    \vspace{2cm}
    
    {\Large\bfseries Grupo 11\par}
    
    \vspace{1.5cm}
    
    \begin{tabular}{rl}
        \textbf{Bruno Ottonelli} & (4.954.242-1) \\
        \textbf{Gabriel Rode}    & (4.535.978-1) \\
    \end{tabular}
    
    \vfill
    
    {\large Universidad de la República\\
    Facultad de Ingeniería\par}
    
    \vspace{0.5cm}
    {\large Junio 2025\par}
\end{titlepage}



% /////////////////////////////////////////////////////////////////
% ///////                       ÍNDICE                      ///////
% /////////////////////////////////////////////////////////////////

\tableofcontents   % Crear el índice



% /////////////////////////////////////////////////////////////////
% ///////                     CONTENIDO                     ///////
% /////////////////////////////////////////////////////////////////
\chapter{Introducción}

En este trabajo presentamos la aplicación práctica del método \textbf{CaDQM} (Context-aware Data Quality Methodology). El objetivo fue, mediante dicha metodología, evaluar y tratar de mejorar la calidad de los datos en un caso particular que simula la fusión de dos librerías, L1 y L2, en una nueva entidad llamada NL.

Durante el desarrollo seguimos las fases definidas por CaDQM: \textit{Data Quality Planning}, \textit{Data Quality Assessment} y \textit{Data Quality Improvement}, lo que nos permitió abordar el trabajo de forma estructurada y ordenada.

En este informe detallamos el proceso seguido, las decisiones que tomamos en cada etapa y cómo la metodología CaDQM nos ayudó a obtener un diagnóstico preciso del estado de calidad de los datos tratados.

\textbf{Nota:} la fase de \textit{Data Quality Improvement} fue aplicada de forma parcial.

\chapter{Fase 1 - Data Quality Planning}

\section{ST1: Elicitation}


\subsection{Entradas y salidas}

\begin{center}
    \resizebox{\textwidth}{!}{ % Ajusta la tabla al ancho de la página
    \begin{tabular}{| p{5cm} | p{10cm} |} % Columnas con ancho ajustado
        \hline
        \multicolumn{2}{|c|}{\cellcolor{blue!30} \textbf{Entradas y salidas}} \\
        \hline
        \rowcolor{blue!15}
        \textbf{Entradas} & \textbf{Salidas} \\
        \hline
        & Base de datos integrada (Data at hand) (\ref{ST1_DataAtHand}) \\
        \hline
        & Reporte con problemas de CD (\ref{ST1_SubsectionProblemasCD}) \\
        \hline
        & Modelo de contexto (\ref{ST1_SubsectionContexto}) \\
        \hline
    \end{tabular}
    }
\end{center}

\noindent CD: Calidad de datos.\\



\subsection{Descripción de la realidad}
El análisis de la calidad de los datos se centrará en información relacionada con libros, proveniente de dos librerías (L1 y L2) que se fusionarán en una nueva librería NL. 

Los encargados de la librería reportan saber de la existencia de muchos problemas de calidad (sin explicitar cuáles) y que éstos se verán potenciados luego de la fusión. Es por esto que la situación genera la necesidad de evaluar la calidad de los datos en la base resultante de la integración de los datos proporcionados por ambas librerías.

Los datos proporcionados por la librería L1 estan distribuidos en dos archivos CSV, uno con detalles de los libros y otros con detalles de las reviews sobre los libros. Ambas tablas se relacionan entre si mediante el atributo \textit{Title}.
\\
\begin{center}
    \resizebox{\textwidth}{!}{ % Ajusta la tabla al ancho de la página
    \begin{tabular}{| p{5cm} | p{10cm} |} % Columnas con ancho ajustado
        \hline
        \multicolumn{2}{|c|}{\cellcolor{blue!30} \textbf{Tablas de la librería L1}} \\
        \hline
        \rowcolor{blue!15}
        \textbf{Nombre de tabla} & \textbf{Atributos} \\
        \hline
        Books\_rating & `Id', `Title', `Price', `User\_id', `profileName', `review/helpfulness', `review/score', `review/time', `review/summary', `review/text' \\
        \hline
        books\_data & `Title', `description', `authors', `image', `previewLink', `publisher', `publishedDate', `infoLink', `categories', `ratingsCount' \\
        \hline
    \end{tabular}
    }
\end{center}

\vspace{1em}

De la librería L2 se obtienen los datos en tres archivos CSV, uno con detalles de los libros, otro con detalle de los usuarios y el último con detalle de las valoraciones de dichos usuarios sobre los libros. Entre ellas se relacionan mediante los atributos \textit{ISBN}, que es un identificador para los libros, y \textit{User-ID}, que es un identificador para los usuarios.

\begin{center}
    \resizebox{\textwidth}{!}{ % Ajusta la tabla al ancho de la página
    \begin{tabular}{| p{5cm} | p{10cm} |} % Columnas con ancho ajustado
        \hline
        \multicolumn{2}{|c|}{\cellcolor{blue!30} \textbf{Tablas de la librería L2}} \\
        \hline
        \rowcolor{blue!15}
        \textbf{Nombre de tabla} & \textbf{Atributos} \\
        \hline
        Books & `ISBN', `Book-Title', `Book-Author', `Year-Of-Publication', `Publisher', `Image-URL-S', `Image-URL-M', `Image-URL-L' \\
        \hline
        Users & `User-ID', `Location', `Age' \\
        \hline
        Ratings & `User-ID', `ISBN', `Book-Rating' \\
        \hline
    \end{tabular}
    }
\end{center}

\subsection{Contexto}
\label{ST1_SubsectionContexto}

\subsubsection{Identificación de componentes del contexto}

\begin{center}
    \resizebox{\textwidth}{!}{ % Ajusta la tabla al ancho de la página
    \begin{tabular}{| p{5cm} | p{10cm} |} % Ajusta el tamaño de las columnas
        \hline % Línea superior
        \multicolumn{2}{|c|}{\cellcolor{blue!30} \textbf{Componentes de Contexto}} \\
        \hline % Línea debajo de la celda "Componentes de Contexto"
        \textbf{Dominio} & D: Libros. \\
        \hline
        \textbf{Fuentes de datos} & Datos obtenidos de ambas librerías que se fusionarán y los proporcionados por el cliente sobre sus realidades. \\
        \hline
        \textbf{Tipos de usuario} & U1: Administrador. \\
                                  & U2: Publicista digital. \\
                                  & U3: Analista de datos. \\
        \hline
        \textbf{Tareas} & T1: Gestión. \\
                        & T2: Análisis. \\
                        & T3: Consulta. \\
        % \hline
        % \textbf{Otros datos?} & Datos relacionados con data at hand. \\
        \hline
        \textbf{Reglas de negocio} & RN1: Cada libro deberá tener asociado un ISBN, un título, al menos un autor y un editor. \\
        \hline
        \textbf{Requerimientos de calidad} & RQ1: Frescura de datos: la base debe actualizarse todos los viernes. \\
                                             & RQ2: Al menos el 80\% de los usuarios que califican los libros deben ser mayores de 18 años. \\
                                             & RQ3: Al menos el 95\% de los libros deben cumplir simultáneamente con los siguientes requisitos: contar con un ISBN, tener el título correctamente escrito y que el nombre del autor incluya al menos un nombre y un apellido. \\
                                             & RQ4: Al menos el 60\% de los libros tengan al menos un score mayor o igual a 5. \\
                                             & RQ5: La librería pretende tener al menos 500 libros y poseer al menos el 20\% de la lista de los 100 mejores libros de Goodreads. \\
                                             & RQ6: Los libros deben contar con fecha de publicación. \\
                                             & RQ7: Los libros deben tener editorial.\\
                                             & RQ8: Los libros deben tener asignado un valor de score.\\
                                             
        \hline
        \textbf{Requerimientos del sistema} & RS1: Los tiempos de respuesta del sitio Web de la NL no pueden superar los 3 segundos. \\
        \hline
        \textbf{Problemas de calidad ya reportados} & Ninguno en particular. \\
        \hline
        \textbf{Necesidades de filtrado} & F1: Libros por fecha (en particular, del año actual). \\
                                          & F2: Libros por editorial. \\
                                          & F3: Top de libros según su score. \\
        \hline
    \end{tabular}
    }
\end{center}

Todos los componentes de contexto surgen de analizar la realidad planteada por el cliente (letra proporcionada para la entrega), sin embargo, los requerimientos RQ6, RQ7 y RQ8 no están detallados explícitamente, sino que surgen de las necesidades de filtrado para que los usuarios puedan realizar correctamente sus labores.

\subsubsection{Contexto}
\begin{center}
    \resizebox{\textwidth}{!}{ % Ajusta la tabla al ancho de la página
    \begin{tabular}{| p{6cm} | p{2cm} | p{2cm} |p{2cm} | p{2cm} |} % Columnas con ancho ajustado
        \hline
        \multicolumn{5}{|c|}{\cellcolor{blue!30} \textbf{Contexto}} \\
        \hline
        \rowcolor{blue!15}
        \textbf{Componente de contexto} & \textbf{Todos los usuarios} & \textbf{U1: Administrador} &  \textbf{U2: Publicista} &  \textbf{U3: Analista}\\
        \hline
        Dominio & D & & &\\
        \hline
        Tareas & T3 & T1 & T2 &\\
        \hline
        Reglas de negocio & RN1 & & &\\
        \hline
        Requerimientos de sistema & RS1 & & &\\
        \hline
        Requerimientos de calidad de datos & RQ5 & RQ6, RQ7, RQ8 & RQ1, RQ2, RQ4 & RQ3\\
        \hline
        Necesidades de filtrado &  & F1, F2, F3 & & \\
        \hline
        Metadatos &   & & &\\
        \hline
        Metadatos de calidad de datos &  & & & \\
        \hline
        Otros datos &  & & & \\
        \hline
    \end{tabular}
    }
\end{center}




\subsection{Primeros problemas de calidad de datos} 
\label{ST1_SubsectionProblemasCD}

Al momento de la integración de la base de datos se identifica como problema de calidad que los datos provenientes de las librerías no contienen los mismos atributos, mencionamos algunos de los ejemplos más importantes:
\begin{itemize}
    \item En el caso de la librería 1 la información referente a libros (\textit{booksData}) no contiene el ISBN que identifica al mismo y solo aparece si los libros poseen una entrada en el libro de ratings (\textit{booksRating}).
    \item  La librería 2 tiene identificado a los usuarios con un entero autogenerado y solo registra la edad y ubicación del mismo mientras que la librería 1 registra mas datos (como el nombre) pero no registra la edad del mismo.
\end{itemize}

\subsection{Data at hand}
\label{ST1_DataAtHand}
El \textit{Data at hand} sobre el que se evaluará la calidad será la base de datos resultante de integrar las bases existentes de ambas librerías. Para ello, se identificaron atributos de ambas bases que representen lo mismo y se ideó una nueva estructura en la que se presentará la información, sin modificar los datos existentes.

La base de datos integrada para la librería NL consta de las tablas y estructura que se muestran en la \autoref{fig:BaseNL}.


% \begin{figure}%[h]
%     \centering
%     \makebox[\textwidth][c]{%
%         \includegraphics[width=\textwidth]{Fase 1/Libros_recortada_2.png}
%     }
%     \caption{Estructura de la base de datos creada}
%     \label{fig:BaseNL}
% \end{figure}

\begin{figure}[htpb]
    \centering
    \makebox[\textwidth][c]{%
        \includegraphics[width=0.9\textwidth]{Fase 1/unificacion.pdf}
    }
    \caption{Estructura de la base de datos creada}
    \label{fig:BaseNL}
\end{figure}



% \newpage

La migración de las tablas de L1 y L2 a NL se hizo como se muestra en las tablas de la presente sección, donde la columna izquierda son los atributos en las tablas originales de L1 y L2, y la columna derecha es en qué atributos se incluyeron de la base de NL.
\\

\begin{table}[htbp]
\centering
\caption{Migración: L1\_bookData → NL\_BOOKS}
\label{Migracion_L1_NL}
\resizebox{\textwidth}{!}{
\renewcommand{\arraystretch}{1.3}
\begin{tabular}{|p{6cm}|p{10cm}|}
    \hline
    \multicolumn{2}{|c|}{\cellcolor{blue!30} \textbf{Books}} \\
    \hline
    \rowcolor{blue!15}
    \textbf{Entradas (L1\_bookData)} & \textbf{Salidas (NL\_BOOKS)} \\
    \hline
    ISBN & NL\_Ratings.ID (relación con BookTitle) \\
    \hline
    Title & BookTitle \\
    \hline
    description & Description \\
    \hline
    authors & AuthorID \\
    \hline
    image & ImageURL\_S \\
    \hline
    previewLink & PreviewLink \\
    \hline
    publisher & PublisherID \\
    \hline
    publishedDate & PublisherDate \\
    \hline
    infoLink & InfoLink \\
    \hline
    categories & Categories \\
    \hline
    ratingsCount & RatingCount \\
    \hline
    Price & Price \\
    \hline
\end{tabular}
}
\end{table}


\begin{table}[htbp]
\centering
\caption{Migración: L1\_booksRating → NL\_USERS}
\label{Migracion_L1Rating_NLUsers}
\resizebox{\textwidth}{!}{
\renewcommand{\arraystretch}{1.3}
\begin{tabular}{|p{6cm}|p{10cm}|}
    \hline
    \multicolumn{2}{|c|}{\cellcolor{blue!30} \textbf{Users}} \\
    \hline
    \rowcolor{blue!15}
    \textbf{Entradas (L1\_bookRating)} & \textbf{Salidas (NL\_USERS)} \\
    \hline
    User\_ID & ID \\
    \hline
    profileName & Name \\
    \hline
\end{tabular}
}
\end{table}


\begin{table}[htbp]
\centering
\caption{Migración: L1\_booksRatings → NL\_RATINGS}
\label{Migracion_L1Ratings_NLRatings}
\resizebox{\textwidth}{!}{
\renewcommand{\arraystretch}{1.3}
\begin{tabular}{|p{6cm}|p{10cm}|}
    \hline
    \multicolumn{2}{|c|}{\cellcolor{blue!30} \textbf{Users}} \\
    \hline
    \rowcolor{blue!15}
    \textbf{Entradas (L1\_booksRatings)} & \textbf{Salidas (NL\_RATINGS)} \\
    \hline
    Title & BookTitle \\
    \hline
    Price & Price \\
    \hline
    User\_id & ID \\
    \hline
    profileName & Name \\
    \hline
    review\_helpfulness & review\_helpfulness \\
    \hline
    review\_score & review\_score \\
    \hline
    review\_time & review\_time \\
    \hline
    review summary & review\_summary \\
    \hline
    review text & review\_text \\
    \hline
\end{tabular}
}
\end{table}



\begin{table}[htbp]
\centering
\caption{Migración: L2\_books → NL\_BOOKS}
\label{Migracion_L2Books_NLBooks}
\resizebox{\textwidth}{!}{
\renewcommand{\arraystretch}{1.3}
\begin{tabular}{|p{6cm}|p{10cm}|}
    \hline
    \multicolumn{2}{|c|}{\cellcolor{blue!30} \textbf{Ratings}} \\
    \hline
    \rowcolor{blue!15}
    \textbf{Entradas (L2\_books)} & \textbf{Salidas (NL\_BOOKS)} \\
    \hline
    ISBN & ISBN \\
    \hline
    Book-Title & BookTitle \\
    \hline
    Book-Author & AuthorID \\
    \hline
    Year-Of-Publication & PublisherDate \\
    \hline
    Publisher & PublisherID \\
    \hline
    Image-URL-S & ImageURL\_S \\
    \hline
    Image-URL-M & ImageURL\_M \\
    \hline
    Image-URL-L & ImageURL\_L \\
    \hline
\end{tabular}
}
\end{table}



\begin{table}[htbp]
\centering
\caption{Migración: L2\_users → NL\_USERS}
\label{Migracion_L2Users_NLUsers}
\resizebox{\textwidth}{!}{
\renewcommand{\arraystretch}{1.3}
\begin{tabular}{|p{6cm}|p{10cm}|}
    \hline
    \multicolumn{2}{|c|}{\cellcolor{blue!30} \textbf{Entradas y salidas}} \\
    \hline
    \rowcolor{blue!15}
    \textbf{Entradas (L2\_users)} & \textbf{Salidas (NL\_USERS)} \\
    \hline
    User-ID & ID \\
    \hline
    Location & Location \\
    \hline
    Age & Age \\
    \hline
\end{tabular}
}
\end{table}





\begin{table}[htbp]
\centering
\caption{Migración: L2\_ratings → NL\_RATINGS}
\label{Migracion_L2Ratings_NLRatings}
\resizebox{\textwidth}{!}{
\renewcommand{\arraystretch}{1.3}
\begin{tabular}{|p{6cm}|p{10cm}|}
    \hline
    \multicolumn{2}{|c|}{\cellcolor{blue!30} \textbf{Entradas y salidas}} \\
    \hline
    \rowcolor{blue!15}
    \textbf{Entradas (L2\_ratings)} & \textbf{Salidas (NL\_RATINGS)} \\
    \hline
    User-ID & User\_ID \\
    \hline
    ISBN & ISBN \\
    \hline
    Book-Rating & review\_score \\
    \hline
\end{tabular}
}
\end{table}


\FloatBarrier



\section{ST2: Data Analysis}

\subsection{Entradas y salidas}

\begin{center}
    \resizebox{\textwidth}{!}{ % Ajusta la tabla al ancho de la página
    \begin{tabular}{| p{5cm} | p{10cm} |} % Columnas con ancho ajustado
        \hline
        \multicolumn{2}{|c|}{\cellcolor{blue!30} \textbf{Entradas y salidas}} \\
        \hline
        \rowcolor{blue!15}
        \textbf{Entradas} & \textbf{Salidas} \\
        \hline
        Base de datos integrada (Data at hand) (\ref{ST1_DataAtHand}) & Reporte de análisis de datos (\ref{ST2_SubsectionAnalisis}) \\
        \hline
        Reporte con problemas de CD (\ref{ST1_SubsectionProblemasCD}) & Reporte con problemas de CD (\autoref{Problemas_ST2}) \\
        \hline
        Modelo de contexto (\ref{ST1_SubsectionContexto}) & Modelo de contexto (\ref{ST2_SubsectionContexto}) \\
        \hline
    \end{tabular}
    }
\end{center}

\noindent CD: Calidad de datos.\\



\subsection{Reporte del análisis de datos}
\label{ST2_SubsectionAnalisis}

\subsubsection{Descripción de las herramientas y técnicas utilizadas para el data profiling}
En el proceso de data profiling, se utilizaron diversas herramientas según cada etapa del flujo de trabajo:
\begin{itemize}
    \item \textit{Python} y \textit{Pandas} para el preprocesamiento, formateo de los archivos en formato .csv y análisis de los datos.
    \item \textit{DBDiagram.io} para el análisis y diseño de la estructura de la base de datos, facilitando su visualización y validación.
    \item \textit{SQL Server} para la creación y gestión de la base de datos, consolidando los datos unificados.
\end{itemize}


% \newpage

\subsubsection{Data profiling}
\label{ST2_DataProfiling}

Para analizar los datos del data at hand, el primer paso fue un estudio estadístico, que se resume en las siguientes tablas:

\begin{figure}[H]
    \centering
    \makebox[\textwidth][c]{%
        \includegraphics[width=\textwidth]{Fase 1/NL_Books.png}
    }
    % \caption{Estructura de la base de datos creada}
    \label{fig:NL_Books}
\end{figure}

\begin{figure}[H]
    \centering
    \makebox[\textwidth][c]{%
        \includegraphics[width=\textwidth]{Fase 1/NL_Users.png}
    }
    % \caption{Estructura de la base de datos creada}
    \label{fig:NL_Users}
\end{figure}

\begin{figure}[H]
    \centering
    \makebox[\textwidth][c]{%
        \includegraphics[width=\textwidth]{Fase 1/NL_Ratings.png}
    }
    % \caption{Estructura de la base de datos creada}
    \label{fig:NL_Ratings}
\end{figure}

En ellas se puede apreciar una gran cantidad de nulos y duplicados. Muchos de estos ya existían previamente en los datos originales proporcionados por ambas librerias, pero, sobre todo en el caso de los nulos, muchos se generaron debido a la integración, ya que ambas bases de datos poseian distintos atributos.

A continuación, se muestra de manera gráfica y porcentual los valores nulos y repetidos:


% \begin{figure}[H]
%     \centering
%     \includegraphics[width=0.85\textwidth]{Fase 1/NL_graf.png}
%     \caption{Gráfica de los valores nulos y repetidos}
%     \label{fig:NLgraf}
% \end{figure}


\begin{figure}[htbp]
    \centering
    \includegraphics[width=\textwidth]{Fase 1/Nulls NL_Users.png}
    \caption{Gráfica de los valores nulos en la tabla \texttt{NL\_Users}}
    \label{fig:NLusers}
\end{figure}

\begin{figure}[htbp]
    \centering
    \includegraphics[width=\textwidth]{Fase 1/Nulls NL_Books.png}
    \caption{Gráfica de los valores nulos en la tabla \texttt{NL\_Books}}
    \label{fig:NLbooks}
\end{figure}

\begin{figure}[htbp]
    \centering
    \includegraphics[width=\textwidth]{Fase 1/Nulls NL_Ratings.png}
    \caption{Gráfica de los valores nulos en la tabla \texttt{NL\_Ratings}}
    \label{fig:NLratings}
\end{figure}

\FloatBarrier




\subsubsection{Listado de los problemas de calidad detectados}
Durante la ejecución del \textit{Data Profiling} se detectaron múltiples errores sobre los datos, los cuales se detallan en la siguiente tabla junto a los ya reportados durante la ST1.

Cabe destacar que, si bien no fueron detectados explícitamente, se consideró oportuno incluir los errores P14 y P15 ya que es de interés del equipo de trabajo medirlos debido a que podrían afectar la calidad y coherencia de los datos.



\begin{table}[H]
\centering
\caption{Problemas detectados en la Calidad de los Datos}
\label{Problemas_ST2}
\resizebox{\textwidth}{!}{
\renewcommand{\arraystretch}{1.3}
\begin{tabular}{|p{5cm}|p{11.5cm}|}
    \hline
    \multicolumn{2}{|c|}{\cellcolor{blue!30} \textbf{Problemas detectados en la Calidad de los Datos}} \\
    \hline
    \rowcolor{blue!15}
    \textbf{Campo} & \textbf{Problema de calidad} \\
    \hline
    NL\_Books.ISBN & P1: Entradas no respetan el formato ISBN ya que algunos están en formato ASID. \\
    \hline
    NL\_Books.PublisherDate & P2: Las fechas tienen distinto formato. \\
    \hline
    NL\_Books.PublisherID & P3: Mismo publisher escrito de forma distinta. \\
    \hline
    NL\_Books.Title & P4: Títulos mal escritos. \\
    \hline
    NL\_Books.AuthorID & P5: Mismo autor escrito de forma distinta. \\
    \hline
    NL\_USERS.Age & P6: Valores de edad poco coherentes (por ejemplo 0). \\
    \hline
    NL\_USERS.Location & P7: Ciudades mal escritas. \\
    \hline
    NL\_RATINGS.review\_time & P8: Formato de fecha/hora inconsistente. \\
    \hline
    NL\_RATINGS.review\_score & P9: Valores no numéricos en la puntuación. \\
    \hline
    NL\_RATINGS.review\_score & P10: Los valores importados entre L1 y L2 manejan distintas escalas (L1 puntúa de 0 a 5 y L2 de 0 a 10). \\
    \hline
    Base de datos & P11: Gran cantidad de nulos en muchos de los atributos. \\
    \hline
    NL\_Books & P12: No hay campo de rating promedio del libro (podría calcularse). \\
    \hline
    NL\_RATINGS.Helpfullness & P13: Este atributo en realidad debería ser dos atributos diferentes: cantidad de votaciones en esa review y cantidad de votaciones que consideraron útil esa review. \\
    \hline
    NL\_Books.AuthorID & P14: Libros indican autores de forma distinta cuando tienen más de uno (dos libros con autores \{A,B\} y \{B,A\} deben considerarse con los mismos autores). \\
    \hline
    NL\_RATINGS.review\_score & P15: Podría haber valores fuera del rango 0 a 10. \\
    % \hline
    % Base de datos & P16: Gran cantidad de nulos en muchos de los atributos, incluyendo claves. \\
    \hline
\end{tabular}
}
\end{table}
















\subsection{Contexto}
\label{ST2_SubsectionContexto}

\subsubsection{Nuevos componentes de contexto}
Dado que se busca a futuro tener una base de datos con un funcionamiento correcto y siguiendo la línea de los problemas de calidad detectados, durante el análisis de datos se identificaron los componentes de contexto RN2, RN3, RN4 y RN5.

Se presenta a continuación la lista completa de componentes de contexto, incluyendo las nuevas componentes mencionadas.

\begin{center}
    \resizebox{\textwidth}{!}{ % Ajusta la tabla al ancho de la página
    \begin{tabular}{| p{3cm} | p{11cm} |} % Ajusta el tamaño de las columnas
        \hline % Línea superior
        \multicolumn{2}{|c|}{\cellcolor{blue!30} \textbf{Componentes de Contexto}} \\
        \hline % Línea debajo de la celda "Componentes de Contexto"
        \textbf{Dominio} & D: Libros. \\
        \hline
        \textbf{Fuentes de datos} & Datos obtenidos de ambas librerías que se fusionarán y los proporcionados por el cliente sobre sus realidades. \\
        \hline
        \textbf{Tipos de usuario} & U1: Administrador. \\
                                  & U2: Publicista digital. \\
                                  & U3: Analista de datos. \\
        \hline
        \textbf{Tareas} & T1: Gestión. \\
                        & T2: Análisis. \\
                        & T3: Consulta. \\
        % \hline
        % \textbf{Otros datos?} & Datos relacionados con data at hand. \\
        \hline
        \textbf{Reglas de negocio}  & RN1: Cada libro deberá tener asociado un ISBN, un título, al menos un autor y un editor. \\
                                    & RN2: El atributo ISBN en NL\_Books debe ser único a cada libro. \\
                                    & RN3: El atributo Price en NL\_Books debe ser un real positivo. \\
                                    & RN4: El atributo Age en NL\_Users debe ser un entero positivo. \\
                                    & RN5: El atributo ID en NL\_Users debe ser único y no vacío. \\
        
        \hline
        \textbf{Requerimientos}              & RQ1: Frescura de datos: la base debe actualizarse todos los viernes. \\
        \textbf{de calidad}                  & RQ2: Al menos el 80\% de los usuarios que califican los libros deben ser mayores de 18 años. \\
                                             & RQ3: Al menos el 95\% de los libros deben cumplir simultáneamente con los siguientes requisitos: contar con un ISBN, tener el título correctamente escrito y que el nombre del autor incluya al menos un nombre y un apellido. \\
                                             & RQ4: Al menos el 60\% de los libros tengan al menos un score mayor o igual a 5. \\
                                             & RQ5: La librería pretende tener al menos 500 libros y poseer al menos el 20\% de la lista de los 100 mejores libros de Goodreads. \\
                                             & RQ6: Los libros deben contar con fecha de publicación. \\
                                             & RQ7: Los libros deben tener editorial.\\
                                             & RQ8: Los libros deben tener asignado un valor de score.\\        
        \hline
        \textbf{Requerimientos del sistema} & RS1: Los tiempos de respuesta del sitio Web de la NL no pueden superar los 3 segundos. \\
        
        \hline
        \textbf{Necesidades de}    & F1: Libros por fecha (en particular, del año actual). \\
        \textbf{filtrado}                   & F2: Libros por editorial. \\
                                            & F3: Top de libros según su score. \\
        \hline
    \end{tabular}
    }
\end{center}




\subsubsection{Contexto}

\begin{center}
    \resizebox{\textwidth}{!}{ % Ajusta la tabla al ancho de la página
    \begin{tabular}{| p{6cm} | p{2cm} | p{2cm} |p{2cm} | p{2cm} |} % Columnas con ancho ajustado
        \hline
        \multicolumn{5}{|c|}{\cellcolor{blue!30} \textbf{Contexto}} \\
        \hline
        \rowcolor{blue!15}
        \textbf{Componente de contexto} & \textbf{Todos los usuarios} & \textbf{U1: Administrador} &  \textbf{U2: Publicista} &  \textbf{U3: Analista}\\
        \hline
        Dominio & D & & &\\
        \hline
        Tareas & T3 & T1 & T2 &\\
        \hline
        Reglas de negocio & RN1, RN2, RN3, RN4, RN5 & & &\\
        \hline
        Requerimientos de sistema & RS1 & & &\\
        \hline
        Requerimientos de calidad de datos & RQ5 & RQ6, RQ7, RQ8 & RQ1, RQ2, RQ4 & RQ3\\
        \hline
        Necesidades de filtrado &  & F1, F2, F3 & & \\
        \hline
        Metadatos &   & & &\\
        \hline
        Metadatos de calidad de datos &  & & & \\
        \hline
        Otros datos &  & & & \\
        \hline
    \end{tabular}
    }
\end{center}




\section{ST3: User requirements analysis}

\subsection{Entradas y salidas}

\begin{center}
    \resizebox{\textwidth}{!}{ % Ajusta la tabla al ancho de la página
    \begin{tabular}{| p{5cm} | p{10cm} |} % Columnas con ancho ajustado
        \hline
        \multicolumn{2}{|c|}{\cellcolor{blue!30} \textbf{Entradas y salidas}} \\
        \hline
        \rowcolor{blue!15}
        \textbf{Entradas} & \textbf{Salidas} \\
        \hline
        Base de datos integrada (Data at hand) (\ref{ST1_DataAtHand}) & Reporte de análisis de requerimientos de usuarios (\ref{ST3_reqUsuarios}) \\
        \hline
        Reporte con problemas de CD (\autoref{Problemas_ST2}) & Reporte con problemas de CD  (\autoref{Problemas_ST2}) \\
        \hline
        Modelo de contexto (\ref{ST2_SubsectionContexto}) & Modelo de contexto (\ref{ST3_SubsectionContexto}) \\
        \hline
    \end{tabular}
    }
\end{center}

\noindent CD: Calidad de datos.\\



\subsection{Reporte de requerimientos de usuario}
\label{ST3_reqUsuarios}

En esta etapa, luego de realizadas consultas al cliente, se identificaron los siguientes requerimientos de calidad.
% RQ9: los nombres de las editoriales deberían estar estandarizados.
% RQ10: Las reglas de formato para nombres (autores, libros, editoriales) son: primera letra del nombre propio en mayúsculas y sin punto al final.
% RQ11: El formato para las fechas será dd/mm/aaaa.

\begin{center}
    \resizebox{\textwidth}{!}{ % Ajusta la tabla al ancho de la página
    \begin{tabular}{| p{5cm} | p{10cm} |} % Ajusta el tamaño de las columnas
        \hline % Línea superior
        \multicolumn{2}{|c|}{\cellcolor{blue!30} \textbf{Nuevas componentes de Contexto}} \\
        \hline % Línea debajo de la celda "Componentes de Contexto"
        \textbf{Requerimientos de calidad}  & RQ9: los nombres de las editoriales deben estar estandarizados.\\
            % \hline
                                    & RQ10: Las reglas de formato para nombres (autores, libros, editoriales) son: primera letra del nombre propio en mayúsculas y sin punto al final.\\
            % \hline
                                    & RQ11: El formato para las fechas será dd/mm/aaaa.\\
            \hline
    \end{tabular}
    }
\end{center}




\subsection{Contexto}
\label{ST3_SubsectionContexto}

\subsubsection{Nuevas componentes de contexto}

Se presenta a continuación la lista completa de componentes de contexto, incluyendo los requerimientos de calidad RQ9, RQ10  y RQ11 identificados durante la ejecución de esta etapa.

\begin{center}
    \resizebox{\textwidth}{!}{ % Ajusta la tabla al ancho de la página
    \begin{tabular}{| p{3cm} | p{11cm} |} % Ajusta el tamaño de las columnas
        \hline % Línea superior
        \multicolumn{2}{|c|}{\cellcolor{blue!30} \textbf{Componentes de Contexto}} \\
        \hline % Línea debajo de la celda "Componentes de Contexto"
        \textbf{Dominio} & D: Libros. \\
        \hline
        \textbf{Fuentes de datos} & Datos obtenidos de ambas librerías que se fusionarán y los proporcionados por el cliente sobre sus realidades. \\
        \hline
        \textbf{Tipos de usuario} & U1: Administrador. \\
                                  & U2: Publicista digital. \\
                                  & U3: Analista de datos. \\
        \hline
        \textbf{Tareas} & T1: Gestión. \\
                        & T2: Análisis. \\
                        & T3: Consulta. \\
        % \hline
        % \textbf{Otros datos?} & Datos relacionados con data at hand. \\
        \hline
        \textbf{Reglas de negocio}  & RN1: Cada libro deberá tener asociado un ISBN, un título, al menos un autor y un editor. \\
                                    & RN2: El atributo ISBN en NL\_Books debe ser único a cada libro. \\
                                    & RN3: El atributo Price en NL\_Books debe ser un real positivo. \\
                                    & RN4: El atributo Age en NL\_Users debe ser un entero positivo. \\
                                    & RN5: El atributo ID en NL\_Users debe ser único y no vacío. \\
        
        \hline
        \textbf{Requerimientos}              & RQ1: Frescura de datos: la base debe actualizarse todos los viernes. \\
        \textbf{de calidad}                  & RQ2: Al menos el 80\% de los usuarios que califican los libros deben ser mayores de 18 años. \\
                                             & RQ3: Al menos el 95\% de los libros deben cumplir simultáneamente con los siguientes requisitos: contar con un ISBN, tener el título correctamente escrito y que el nombre del autor incluya al menos un nombre y un apellido. \\
                                             & RQ4: Al menos el 60\% de los libros tengan al menos un score mayor o igual a 5. \\
                                             & RQ5: La librería pretende tener al menos 500 libros y poseer al menos el 20\% de la lista de los 100 mejores libros de Goodreads. \\
                                             & RQ6: Los libros deben contar con fecha de publicación. \\
                                             & RQ7: Los libros deben tener editorial.\\
                                             & RQ8: Los libros deben tener asignado un valor de score.\\
                                             & RQ9: los nombres de las editoriales deben estar estandarizados.\\
                                             & RQ10: Las reglas de formato para nombres (autores, libros, editoriales) son: primera letra del nombre propio en mayúsculas y sin punto al final.\\
                                             & RQ11: El formato para las fechas será dd/mm/aaaa.\\
                                             
        \hline
        \textbf{Requerimientos del sistema} & RS1: Los tiempos de respuesta del sitio Web de la NL no pueden superar los 3 segundos. \\
        
        \hline
        \textbf{Necesidades de}    & F1: Libros por fecha (en particular, del año actual). \\
        \textbf{filtrado}                   & F2: Libros por editorial. \\
                                            & F3: Top de libros según su score. \\
        \hline
    \end{tabular}
    }
\end{center}



\subsubsection{Contexto}
\begin{center}
    \resizebox{\textwidth}{!}{ % Ajusta la tabla al ancho de la página
    \begin{tabular}{| p{6cm} | p{2cm} | p{2cm} |p{2cm} | p{2cm} |} % Columnas con ancho ajustado
        \hline
        \multicolumn{5}{|c|}{\cellcolor{blue!30} \textbf{Contexto}} \\
        \hline
        \rowcolor{blue!15}
        \textbf{Componente de contexto} & \textbf{Todos los usuarios} & \textbf{U1: Administrador} &  \textbf{U2: Publicista} &  \textbf{U3: Analista}\\
        \hline
        Dominio & D & & &\\
        \hline
        Tareas & T3 & T1 & T2 &\\
        \hline
        Reglas de negocio & RN1, RN2, RN3, RN4, RN5 & & &\\
        \hline
        Requerimientos de sistema & RS1 & & &\\
        \hline
        Requerimientos de calidad de datos & RQ5, RQ9, RQ10, RQ11 & RQ6, RQ7, RQ8 & RQ1, RQ2, RQ4 & RQ3\\
        \hline
        Necesidades de filtrado &  & F1, F2, F3 & & \\
        \hline
        Metadatos &   & & &\\
        \hline
        Metadatos de calidad de datos &  & & & \\
        \hline
        Otros datos &  & & & \\
        \hline
    \end{tabular}
    }
\end{center}




\section{Entradas y salidas de la Fase 1 completa}
Como preparación para abordar la Fase 2 del modelo CaDQM, se presentan a continuación las entradas y salidas generales de la Fase 1.

\begin{center}
    \resizebox{\textwidth}{!}{ % Ajusta la tabla al ancho de la página
    \begin{tabular}{| p{5cm} | p{10cm} |} % Columnas con ancho ajustado
        \hline
        \multicolumn{2}{|c|}{\cellcolor{blue!30} \textbf{Entradas y salidas}} \\
        \hline
        \rowcolor{blue!15}
        \textbf{Entradas} & \textbf{Salidas} \\
        \hline
        & Base de datos integrada (Data at hand) (\ref{ST1_DataAtHand}) \\
        \hline
         & Reporte de análisis de datos (\ref{ST2_SubsectionAnalisis}) \\
        \hline
         & Reporte de análisis de requerimientos de usuarios (\ref{ST3_reqUsuarios}) \\
        \hline
         & Reporte con problemas de CD  (\autoref{Problemas_ST2}) \\
        \hline
         & Modelo de contexto (\ref{ST3_SubsectionContexto}) \\
        \hline
    \end{tabular}
    }
\end{center}

\noindent CD: Calidad de datos.\\



\section{Descripción del desarrollo}

% Se identifico el contexto en la realidad planteada
% Se leyo la documentacion sobre las bases de datos proporcionadas
% Se modificaron los csv para poder ser aiertos por sqlserver
% Se integraron las bases para realizar el data at hand
% Se utilizo Python y Pandas para el analisis de las tablas
% se identificaron nuevos requerimientos de calidad.



Siguiendo la metodología CaQDM, el primer paso fue identificar el contexto de la realidad planteada, identificando los elementos de contexto en ella para armar luego el modelo de contexto en base a estos. Luego se revisó la documentación disponible sobre las bases de datos proporcionadas, para entender su estructura y contenido, pero sin indagar en los datos.

Dado que los archivos estaban en formato CSV y su contenido tenia ciertas particularidades, fue necesario realizar algunas modificaciones tales como cambiar el caracter separador y ver que el contenido de las celdas estaba entre comillas, para que pudieran abrirse correctamente en \textit{SQL Server}. Una vez adaptados y cargados correctamente, se diseñó la nueva estructura de la base de datos de NL donde se integraron las distintas fuentes de datos para conformar el data at hand.

Para la realización del proceso, el análisis de las tablas se realizó utilizando \textit{Python} y la biblioteca \textit{Pandas}, lo que permitió explorar y procesar los datos de manera rápida y sencilla. Para el diseño de la nueva base, se utilizaron herramientas visuales como \textit{DBDiagram.io}, que luego se implementarían en \textit{SQL server}.

\section{Conclusiones de la fase 1}

%problemas para integrar la base
% luego Costo entender los limites de cada etapa y el contenido, pero lo logramos y se sortearon con exito los obstaculos encontrados
% Logramos aplicar las distintas etapas de la fase 1 de CaQDM

La implementación de la primer fase de CaQDM en este proyecto fue una experiencia interesante en el manejo de datos donde pudimos aplicar el conocimiento y la metodología vistos en clase.

 Uno de los principales retos fue la migración de la información proveniente de los conjuntos \textbf{L1} y \textbf{L2}, los cuales estaban originalmente en archivos \textbf{csv} y, en el caso particular de uno de ellos, tenia un tamaño grande que hizo difícil su tratamiento por lo que fue necesario una etapa previa de tratamiento de los datos, además de particularidades en la forma que estaban almacenados los datos, que generaron complicaciones a la hora de importar. No obstante, estos problemas pudieron ser sorteados con éxito.

Por otro lado, al ser nuestra primer experiencia aplicando el modelo sobre un caso tan particular como este, donde se debian integrar datos, tuvimos ciertas dificultades al inicio en delimitar el alcance de cada una de las etapas del modelo, pero que luego fueron aclaradas y ejecutadas correctamente.

Finalmente consideramos un logro importante el haber podido aplicar de forma satisfactoria la \textbf{Fase 1 del marco CaQDM} (Calidad y Gestión de Datos), aplicando en un caso con información real.

 
\chapter{Fase 2 - Data Quality Assessment}

\section{ST4: DQ Model Definition}

\subsection{Entradas y salidas}
\begin{center}
    \resizebox{\textwidth}{!}{ % Ajusta la tabla al ancho de la página
    \begin{tabular}{| p{8cm} | p{8cm} |} % Columnas con ancho ajustado
        \hline
        \multicolumn{2}{|c|}{\cellcolor{blue!30} \textbf{Entradas y salidas}} \\
        \hline
        \rowcolor{blue!15}
        \textbf{Entradas} & \textbf{Salidas} \\
        \hline
        Reporte del análisis de requerimientos de usuarios (\ref{ST3_reqUsuarios})& Reporte de problemas de CD priorizados (\ref{ST4_problemas_priorizados})\\
        \hline
        Reporte del análisis de datos (\ref{ST2_SubsectionAnalisis}) & Modelo de CD (\ref{ST4_modelo_calidad_contextual})\\
        \hline
        Reporte de problemas de CD (\autoref{Problemas_ST2}) &  \\
        \hline
        Modelo de Contexto (\ref{ST3_SubsectionContexto}) & \\
        \hline
    \end{tabular}
    }
\end{center}

\noindent CD: Calidad de datos.\\



\subsection{Priorización de los problemas de calidad de datos}
\label{ST4_problemas_priorizados}

La siguiente tabla muestra los problemas de calidad de datos detectados durante la \textit{Fase 1}, ya priorizados.

Para la priorización, se consideró que todos los problemas de calidad asociados a reglas de negocio tuvieran alta prioridad, dado que representan un interés central para el cliente o constituyen requisitos esenciales para el correcto funcionamiento de cualquier sistema de bases de datos.

Los errores de formato o que tienen que ver con una posible futura reestructuración de la base de datos fueron clasificados con una prioridad media, ya que, si bien no impiden el uso de los datos, pueden dificultar su utilización,  presentación o interpretación.

Por otro lado, aquellos problemas relacionados con errores ortográficos, tipográficos o inconsistencias en la escritura se asignaron con prioridad baja. La única excepción a esta regla es el caso del \textit{P3} con prioridad media, ya que un error en el \textit{Publisher} interfiere directamente con la necesidad de filtrado \textit{F2}, imposibilitando parte de las tareas del usuario \textit{U1}.


\begin{center}
\resizebox{\textwidth}{!}{
\renewcommand{\arraystretch}{1.3} % Aumenta el espacio entre filas
\begin{tabular}{|p{5cm}|p{8.5cm}|p{2cm}|}
    \hline
    \multicolumn{3}{|c|}{\cellcolor{blue!30} \textbf{Problemas detectados en la Calidad de los Datos}} \\
    \hline
    \rowcolor{blue!15}
    \textbf{Campo} & \textbf{Problema de calidad} & \textbf{Prioridad} \\
    \hline
    NL\_Books.ISBN & P1: Entradas no respetan el formato ISBN ya que algunos están en formato ASID. & Alta \\
    \hline
    NL\_Books.PublisherDate & P2: Las fechas tienen distinto formato. & Media \\
    \hline
    NL\_Books.PublisherID & P3: Mismo publisher escrito de forma distinta. & Media \\
    \hline
    NL\_Books.Title & P4: Títulos mal escritos. & Baja \\
    \hline
    NL\_Books.AuthorID & P5: Mismo autor escrito de forma distinta. & Baja \\
    \hline
    NL\_USERS.Age & P6: Valores de edad poco coherentes (por ejemplo 0). & Alta \\
    \hline
    NL\_USERS.Location & P7: Ciudades mal escritas. & Baja \\
    \hline
    NL\_RATINGS.review\_time & P8: Formato de fecha/hora inconsistente. & Media \\
    \hline
    NL\_RATINGS.review\_score & P9: Valores no numéricos en la puntuación. & Media \\
    \hline
    NL\_RATINGS.review\_score & P10: Los valores importados entre L1 y L2 manejan distintas escalas (L1 puntúa de 0 a 5 y L2 de 0 a 10). & Media \\
    \hline
    Base de datos & P11: Gran cantidad de nulos en muchos de los atributos, incluyendo claves. & Alta \\
    \hline
    NL\_Books & P12: No hay campo de rating promedio del libro (podría calcularse). & Media
    \\
    \hline
    NL\_RATINGS.Helpfullness & P13: Este atributo en realidad debería ser dos atributos diferentes: cantidad de votaciones en esa review y cantidad de votaciones que consideraron útil esa review. & Media
    \\
    \hline
    NL\_Books.AuthorID & P14: Libros indican autores de forma distinta cuando tienen mas de uno (dos libros con autores \{A,B\} y \{B,A\} deben considerarse con los mismos autores). & Baja 
    \\
    \hline
     NL\_RATINGS.review\_score & P15: Podría haber valores fueron del rango 0 a 10. &  Media
    \\
    \hline
\end{tabular}
}
\end{center}







\subsection{Métricas}

\subsubsection{check\_ISBN}

\begin{center}
\resizebox{\textwidth}{!}{
\begin{tabular}{| p{3.5cm} | p{11.5cm} |}
    \hline
    \multicolumn{2}{|c|}{\cellcolor{blue!30} \textbf{Métrica}} \\
    \hline
    \textbf{Nombre} & \textbf{check\_ISBN} \\
    \hline
    \textbf{Descripción} & Controla si el valor es un ISBN válido. \\
    \hline
    \textbf{Granularidad} & Celda \\
    \hline
    \textbf{Dominio del Resultado} & \{0, 1\} \\
    \hline
\end{tabular}
}
\end{center}

% \vspace{1em}

\begin{center}
\resizebox{\textwidth}{!}{
\begin{tabular}{| p{3.5cm} | p{11.5cm} |}
    \hline
    \multicolumn{2}{|c|}{\cellcolor{blue!30} \textbf{Método}} \\
    \hline
    \textbf{Descripción} & Implementa la métrica Check\_ISBN teniendo en cuenta la estructura de un código ISBN. \\
    \hline
    \textbf{Tipos de datos de entrada} & String \\
    \hline
    \textbf{Tipos de datos de salida} & Boolean \\
    \hline
    \textbf{Algoritmo} & 
\texttt{def Check\_ISBN(codigo):} \\ 
& \texttt{\ \ size = len(codigo)} \\
& \texttt{\ \ \# Chequea para el caso de un ISBN10} \\
& \texttt{\ \ if (size == 10):} \\
& \texttt{\ \ \ \ verifica que los primeros 9 caracteres sean dígitos} \\
& \texttt{\ \ \ \ verifica que el último caracter sea un dígito o una "X"} \\
& \texttt{\ \ \ \ verifica que el dígito verificador sea correcto} \\
& \texttt{\ \ \# Chequea para el caso de un ISBN13} \\
& \texttt{\ \ if (size == 13):} \\
& \texttt{\ \ \ \ verifica que todos los caracteres sean dígitos} \\
& \texttt{\ \ \ \ verifica que el dígito verificador sea correcto} \\
    \hline
\end{tabular}
}
\end{center}

% \vspace{1em}

\begin{center}
\resizebox{\textwidth}{!}{
\begin{tabular}{| p{3.5cm} | p{11.5cm} |}
    \hline
    \multicolumn{2}{|c|}{\cellcolor{blue!30} \textbf{Método aplicado}} \\
    \hline
    \textbf{Tipo} & Medición \\
    \hline
    \textbf{Descripción} & Utiliza el algoritmo de cálculo de ISBN para verificar la validez del dato. \\
    \hline
    \textbf{Aplicado a} & Atributo ⟪ISBN⟫ de la tabla \texttt{NL\_Books} \\
    \hline
\end{tabular}
}
\end{center}




% -------------

\subsubsection{check\_edades}

\begin{center}
\resizebox{\textwidth}{!}{
\begin{tabular}{|p{3.5cm} | p{11.5cm}|}
\hline
\multicolumn{2}{|c|}{\cellcolor{blue!30} \textbf{Métrica}} \\
\hline
\textbf{Nombre} & check\_edades \\
\hline
\textbf{Descripción} & Controla si la edad es válida. \\
\hline
\textbf{Granularidad} & Celda \\
\hline
\textbf{Dominio del Resultado} & \{0, 1\} \\
\hline
\end{tabular}
}
\end{center}


\begin{center}
\resizebox{\textwidth}{!}{
\begin{tabular}{|p{3.5cm} | p{11.5cm}|}
\hline
\multicolumn{2}{|c|}{\cellcolor{blue!30} \textbf{Método}} \\
\hline
\textbf{Descripción} & Implementa la métrica check\_edades verificando si la edad tiene formato correcto y es un valor razonable. \\
\hline
\textbf{Tipos de datos de entrada} & String \\
\hline
\textbf{Tipos de datos de salida} & Boolean \\
\hline
\textbf{Algoritmo} & 
\texttt{IsNumeric(edad) AND edad > 0 AND edad < 100} \\
\hline
\end{tabular}
}
\end{center}



\begin{center}
\resizebox{\textwidth}{!}{
\begin{tabular}{|p{3.5cm} | p{11.5cm}|}
\hline
\multicolumn{2}{|c|}{\cellcolor{blue!30} \textbf{Método aplicado}} \\
\hline
\textbf{Tipo} & Medición \\
\hline
\textbf{Descripción} & Verifica que tenga un valor numérico entre 1 y 99. \\
\hline
\textbf{Aplicado a} & Atributo ⟪Age⟫ de la tabla \texttt{NL\_Users} \\
\hline
\end{tabular}
}
\end{center}


% -------------------------

\subsubsection{check\_price}

\begin{center}
\resizebox{\textwidth}{!}{
\begin{tabular}{|p{3.5cm} | p{11.5cm}|}
\hline
\multicolumn{2}{|c|}{\cellcolor{blue!30} \textbf{Métrica}} \\
\hline
\textbf{Nombre} & check\_price \\
\hline
\textbf{Descripción} & Controla si el precio es válido. \\
\hline
\textbf{Granularidad} & Celda \\
\hline
\textbf{Dominio del Resultado} & \{0, 1\} \\
\hline
\end{tabular}
}
\end{center}


\begin{center}
\resizebox{\textwidth}{!}{
\begin{tabular}{|p{3.5cm} | p{11.5cm}|}
\hline
\multicolumn{2}{|c|}{\cellcolor{blue!30} \textbf{Método}} \\
\hline
\textbf{Descripción} & Implementa la métrica check\_price verificando si el valor tiene formato correcto. \\
\hline
\textbf{Tipos de datos de entrada} & String \\
\hline
\textbf{Tipos de datos de salida} & Boolean \\
\hline
\textbf{Algoritmo} & \texttt{IsNumeric(price) AND price >= 0} \\
\hline
\end{tabular}
}
\end{center}


\begin{center}
\resizebox{\textwidth}{!}{
\begin{tabular}{|p{3.5cm} | p{11.5cm}|}
\hline
\multicolumn{2}{|c|}{\cellcolor{blue!30} \textbf{Método aplicado}} \\
\hline
\textbf{Tipo} & Medición \\
\hline
\textbf{Descripción} & Verifica que tenga un valor numérico positivo con una consulta SQL. \\
\hline
\textbf{Aplicado a} & Atributo ⟪Price⟫ de la tabla \texttt{NL\_Books} \\
\hline
\end{tabular}
}
\end{center}


% ---------------------------



\subsubsection{duplicate\_ratio}

\begin{center}
\resizebox{\textwidth}{!}{
\begin{tabular}{|p{3.5cm} | p{11.5cm}|}
\hline
\multicolumn{2}{|c|}{\cellcolor{blue!30} \textbf{Métrica}} \\
\hline
\textbf{Nombre} & duplicate\_ratio \\
\hline
\textbf{Descripción} & Da el porcentaje de valores duplicados. \\
\hline
\textbf{Granularidad} & Columna \\
\hline
\textbf{Dominio del Resultado} & [0...1] \\
\hline
\end{tabular}
}
\end{center}


\begin{center}
\resizebox{\textwidth}{!}{
\begin{tabular}{|p{3.5cm} | p{11.5cm}|}
\hline
\multicolumn{2}{|c|}{\cellcolor{blue!30} \textbf{Método}} \\
\hline
\textbf{Descripción} & Implementa la métrica duplicate\_ratio sobre un atributo dado. \\
\hline
\textbf{Tipos de datos de entrada} & String \\
\hline
\textbf{Tipos de datos de salida} & Float \\
\hline
\textbf{Algoritmo} & 
\texttt{def duplicate\_ratio(data, column):} \\ 
& \texttt{\ \ \ \ total = len(data)} \\
& \texttt{\ \ \ \ duplicated = data.duplicated(subset=[column]).sum()} \\
& \texttt{\ \ \ \ return duplicated / total} \\
\hline
\end{tabular}
}
\end{center}


\begin{center}
\resizebox{\textwidth}{!}{
\begin{tabular}{|p{3.5cm} | p{11.5cm}|}
\hline
\multicolumn{2}{|c|}{\cellcolor{blue!30} \textbf{Método aplicado}} \\
\hline
\textbf{Tipo} & Medición \\
\hline
\textbf{Descripción} & Dados los datos y el atributo, calcula el porcentaje de valores duplicados en ese atributo. \\
\hline
\textbf{Aplicado a} & Atributos ⟪ISBN⟫, ⟪AuthorID⟫, ⟪PublisherID⟫, ⟪User\_ID⟫ de la tabla \texttt{NL\_Books}. \newline
Atributo ⟪ID⟫ de la tabla \texttt{NL\_Users}. \\
\hline
\end{tabular}
}
\end{center}


% ---------------------------------------



\subsubsection{check\_RN1}

\begin{center}
\resizebox{\textwidth}{!}{
\begin{tabular}{|p{3.5cm} | p{11.5cm}|}
\hline
\multicolumn{2}{|c|}{\cellcolor{blue!30} \textbf{Métrica}} \\
\hline
\textbf{Nombre} & check\_RN1 \\
\hline
\textbf{Descripción} & Da el porcentaje de entradas de la tabla que tienen al menos un campo vacío entre los atributos \texttt{isbn}, \texttt{titulo}, \texttt{autor} y \texttt{editor}. \\
\hline
\textbf{Granularidad} & Conjunto de columnas \\
\hline
\textbf{Dominio del Resultado} & [0...1] \\
\hline
\end{tabular}
}
\end{center}


\begin{center}
\resizebox{\textwidth}{!}{
\begin{tabular}{|p{3.5cm} | p{11.5cm}|}
\hline
\multicolumn{2}{|c|}{\cellcolor{blue!30} \textbf{Método}} \\
\hline
\textbf{Descripción} & Implementa la métrica check\_RN1 sobre una tabla dada. \\
\hline
\textbf{Tipos de datos de entrada} & String \\
\hline
\textbf{Tipos de datos de salida} & Float \\
\hline
\textbf{Algoritmo} & 
\texttt{def Check\_RN1(datos):} \\
& \texttt{\ \ \ \ \# Seleccionar solo las columnas relevantes (isbn, titulo, autor, editor)} \\
& \texttt{\ \ \ \ datos\_relevantes = datos[['isbn', 'titulo', 'autor', 'editor']]} \\
& \texttt{\ \ \ \ \# Verificar cuántas filas tienen al menos un campo NULL entre estas columnas} \\
& \texttt{\ \ \ \ incompletos = datos\_relevantes.isnull().any(axis=1).sum()} \\
& \texttt{\ \ \ \ \# Calcular la proporción de filas incompletas} \\
& \texttt{\ \ \ \ return incompletos / len(datos)} \\
\hline
\end{tabular}
}
\end{center}


\begin{center}
\resizebox{\textwidth}{!}{
\begin{tabular}{|p{3.5cm} | p{11.5cm}|}
\hline
\multicolumn{2}{|c|}{\cellcolor{blue!30} \textbf{Método aplicado}} \\
\hline
\textbf{Tipo} & Medición \\
\hline
\textbf{Descripción} & Dada una tabla, calcula el porcentaje de entradas que tienen al menos un campo vacío entre los atributos \texttt{isbn}, \texttt{titulo}, \texttt{autor} y \texttt{editor}. \\
\hline
\textbf{Aplicado a} & Tabla \texttt{NL\_Books}. \\
\hline
\end{tabular}
}
\end{center}



% -----------------------------



\subsubsection{contar\_nulls}

\begin{center}
\resizebox{\textwidth}{!}{
\begin{tabular}{|p{3.5cm} | p{11.5cm}|}
\hline
\multicolumn{2}{|c|}{\cellcolor{blue!30} \textbf{Métrica}} \\
\hline
\textbf{Nombre} & contar\_nulls \\
\hline
\textbf{Descripción} & Calcula el porcentaje de entradas vacías en una columna. \\
\hline
\textbf{Granularidad} & Columna \\
\hline
\textbf{Dominio del Resultado} & [0...1] \\
\hline
\end{tabular}
}
\end{center}

\begin{center}
\resizebox{\textwidth}{!}{
\begin{tabular}{|p{3.5cm} | p{11.5cm}|}
\hline
\multicolumn{2}{|c|}{\cellcolor{blue!30} \textbf{Método}} \\
\hline
\textbf{Descripción} & Implementa la métrica contar\_nulls. \\
\hline
\textbf{Tipos de datos de entrada} & String \\
\hline
\textbf{Tipos de datos de salida} & Float \\
\hline
\textbf{Algoritmo} & \texttt{def porcentaje\_nulos(atributo):} \\
& \texttt{\ \ \ \ nulos = atributo.isnull()} \\
& \texttt{\ \ \ \ contador\_nulos = nulos.sum()} \\
& \texttt{\ \ \ \ porcentaje = contador\_nulos / len(atributo)} \\
& \texttt{\ \ \ \ return porcentaje} \\
\hline
\end{tabular}
}
\end{center}

\begin{center}
\resizebox{\textwidth}{!}{
\begin{tabular}{|p{3.5cm} | p{11.5cm}|}
\hline
\multicolumn{2}{|c|}{\cellcolor{blue!30} \textbf{Método aplicado}} \\
\hline
\textbf{Tipo} & Medición \\
\hline
\textbf{Descripción} & Dado un atributo de una tabla, calcula el porcentaje de entradas vacías. \\
\hline
\textbf{Aplicado a} & Cualquier atributo de cualquier tabla. \\
\hline
\end{tabular}
}
\end{center}


% ---------------------------------------

\subsubsection{concistencia\_ratings}

\begin{center}
\resizebox{\textwidth}{!}{
\begin{tabular}{|p{3.5cm} | p{11.5cm}|}
\hline
\multicolumn{2}{|c|}{\cellcolor{blue!30} \textbf{Métrica}} \\
\hline
\textbf{Nombre} & concistencia\_ratings \\
\hline
\textbf{Descripción} & Verifica que la cantidad de ratings entre las distintas tablas sea coherente. \\
\hline
\textbf{Granularidad} & Conjunto de columnas \\
\hline
\textbf{Dominio del Resultado} & \{0, 1\} \\
\hline
\end{tabular}
}
\end{center}

\begin{center}
\resizebox{\textwidth}{!}{
\begin{tabular}{|p{3.5cm} | p{11.5cm}|}
\hline
\multicolumn{2}{|c|}{\cellcolor{blue!30} \textbf{Método}} \\
\hline
\textbf{Descripción} & Implementa la métrica concistencia\_ratings contando la cantidad de ocurrencias de un ISBN válido en la tabla NL\_ratings y lo compara con lo declarado en el atributo rating\_count de NL\_Books. \\
\hline
\textbf{Tipos de datos de entrada} & String \\
\hline
\textbf{Tipos de datos de salida} & Boolean \\
\hline
\textbf{Algoritmo} & \texttt{def funcion\_consistencia\_ratings(libros, valid\_isbn, ratings):} \\
& \texttt{\ \ \ \ libros\_validos = libros[valid\_isbn==True]} \\
& \texttt{\ \ \ \ for libro in libros\_validos:} \\
& \texttt{\ \ \ \ \ \ \ \ isbn = libro['isbn']} \\
& \texttt{\ \ \ \ \ \ \ \ cantidad = (ratings['isbn'] == isbn).sum()} \\
& \texttt{\ \ \ \ \ \ \ \ contados[isbn] = cantidad} \\
& \texttt{\ \ \ \ rating\_esperado = libros\_validos['rating\_counts']} \\
& \texttt{\ \ \ \ return rating\_esperado == contados} \\
\hline
\end{tabular}
}
\end{center}

\begin{center}
\resizebox{\textwidth}{!}{
\begin{tabular}{|p{3.5cm} | p{11.5cm}|}
\hline
\multicolumn{2}{|c|}{\cellcolor{blue!30} \textbf{Método aplicado}} \\
\hline
\textbf{Tipo} & Agregación \\
\hline
\textbf{Descripción} & Dada una tabla de reviews, una tabla de libros (ambas con el campo isbn) y una lista que indiquen si son válidos o no, determina si hay congruencia entre lo registrado en ambas tablas sobre los ratings. \\
\hline
\textbf{Aplicado a} & Conjunto de atributos (NL\_ratings.ISBN, NL\_books.ISBN, NL\_books.Rating\_Count) \\
\hline
\end{tabular}
}
\end{center}


% ------------------------------------------


\subsubsection{consistencia\_fechas}

\begin{center}
\resizebox{\textwidth}{!}{
\begin{tabular}{|p{3.5cm} | p{11.5cm}|}
\hline
\multicolumn{2}{|c|}{\cellcolor{blue!30} \textbf{Métrica}} \\
\hline
\textbf{Nombre} & consistencia\_fechas \\
\hline
\textbf{Descripción} & Verifica que la fecha de un rating sea posterior a la fecha de publicación del libro. \\
\hline
\textbf{Granularidad} &  Conjunto de columnas \\
\hline
\textbf{Dominio del Resultado} & \{0, 1\} \\
\hline
\end{tabular}
}
\end{center}

\begin{center}
\resizebox{\textwidth}{!}{
\begin{tabular}{|p{3.5cm} | p{11.5cm}|}
\hline
\multicolumn{2}{|c|}{\cellcolor{blue!30} \textbf{Método}} \\
\hline
\textbf{Descripción} & Implementa la métrica concistencia\_fechas comparando las fechas de review\_time de la tabla NL\_reviews y la de PublisherDate de NL\_Books. \\
\hline
\textbf{Tipos de datos de entrada} & String \\
\hline
\textbf{Tipos de datos de salida} & Boolean \\
\hline
\textbf{Algoritmo} & \texttt{def consistencia\_fechas(NL\_reviews, NL\_books):} \\
& \texttt{\ \ \ \ result = []} \\
& \texttt{\ \ \ \ for i in range(len(NL\_reviews)):\ } \\
& \texttt{\ \ \ \ \ \ \ \ isbn = NL\_reviews['isbn'][i]} \\
& \texttt{\ \ \ \ \ \ \ \ fila\_libro = NL\_books[NL\_books['isbn'] == isbn]} \\
& \texttt{\ \ \ \ \ \ \ \ review\_time, PublishedDate = fila\_libro[['review\_time', 'PublishedDate']].values[0]} \\
& \texttt{\ \ \ \ \ \ \ \ result.append(review\_time >= PublishedDate)} \\
& \texttt{\ \ \ \ return result} \\
\hline
\end{tabular}
}
\end{center}

\begin{center}
\resizebox{\textwidth}{!}{
\begin{tabular}{|p{3.5cm} | p{11.5cm}|}
\hline
\multicolumn{2}{|c|}{\cellcolor{blue!30} \textbf{Método aplicado}} \\
\hline
\textbf{Tipo} & Medición \\
\hline
\textbf{Descripción} & Dada una tabla de reviews y una tabla de libros (ambas con el campo isbn), determina si la fecha de la review es posterior a la publicación del libro. \\
\hline
\textbf{Aplicado a} & Conjunto de atributos ⟪NL_Reviews.review_time, NL_Books.PublisherDate⟫  \\
\hline
\end{tabular}
}
\end{center}


% -------------------------------



\subsubsection{missing\_rating\_books}

\begin{center}
\resizebox{\textwidth}{!}{
\begin{tabular}{|p{3.5cm} | p{11.5cm}|}
\hline
\multicolumn{2}{|c|}{\cellcolor{blue!30} \textbf{Métrica}} \\
\hline
\textbf{Nombre} & missing\_rating\_books \\
\hline
\textbf{Descripción} & Para ratings sobre libros, controla la existencia de estos en la base de datos. \\
\hline
\textbf{Granularidad} & Celda \\
\hline
\textbf{Dominio del Resultado} & \{0, 1\} \\
\hline
\end{tabular}
}
\end{center}

\begin{center}
\resizebox{\textwidth}{!}{
\begin{tabular}{|p{3.5cm} | p{11.5cm}|}
\hline
\multicolumn{2}{|c|}{\cellcolor{blue!30} \textbf{Método}} \\
\hline
\textbf{Descripción} & Implementa la métrica check\_rating\_books para ver si los libros indicados en los ratings existen. \\
\hline
\textbf{Tipos de datos de entrada} & String \\
\hline
\textbf{Tipos de datos de salida} & Boolean \\
\hline
\textbf{Algoritmo} & \texttt{SELECT COUNT(*) FROM NL\_Ratings WHERE NOT EXISTS(SELECT * FROM NL\_Books WHERE NL\_Ratings.ISBN = NL\_Books.ISBN)} \\
\hline
\end{tabular}
}
\end{center}

\begin{center}
\resizebox{\textwidth}{!}{
\begin{tabular}{|p{3.5cm} | p{11.5cm}|}
\hline
\multicolumn{2}{|c|}{\cellcolor{blue!30} \textbf{Método aplicado}} \\
\hline
\textbf{Tipo} & Medición \\
\hline
\textbf{Descripción} & Verifica que el libro en NL\_Ratings exista en NL\_Books \\
\hline
\textbf{Aplicado a} & Atributos ⟪ISBN⟫ en las tablas NL\_Books y NL\_Ratings \\
\hline
\end{tabular}
}
\end{center}


% ------------------------------------

\subsubsection{date\_format}

\begin{center}
\resizebox{\textwidth}{!}{
\begin{tabular}{|p{3.5cm} | p{11.5cm}|}
\hline
\multicolumn{2}{|c|}{\cellcolor{blue!30} \textbf{Métrica}} \\
\hline
\textbf{Nombre} & date\_format \\
\hline
\textbf{Descripción} & Controla el formato de fecha de todas las columnas tipo fecha \\
\hline
\textbf{Granularidad} & Celda \\
\hline
\textbf{Dominio del Resultado} & \{0, 1\} \\
\hline
\end{tabular}
}
\end{center}

\begin{center}
\resizebox{\textwidth}{!}{
\begin{tabular}{|p{3.5cm} | p{11.5cm}|}
\hline
\multicolumn{2}{|c|}{\cellcolor{blue!30} \textbf{Método}} \\
\hline
\textbf{Descripción} & Implementa la métrica date\_format para controlar que el formato de fecha sea correcto \\
\hline
\textbf{Tipos de datos de entrada} & String \\
\hline
\textbf{Tipos de datos de salida} & Boolean \\
\hline
\textbf{Algoritmo} & \texttt{TRY\_CONVERT(DATE,date,103)} \\
\hline
\end{tabular}
}
\end{center}

\begin{center}
\resizebox{\textwidth}{!}{
\begin{tabular}{|p{3.5cm} | p{11.5cm}|}
\hline
\multicolumn{2}{|c|}{\cellcolor{blue!30} \textbf{Método aplicado}} \\
\hline
\textbf{Tipo} & Medición \\
\hline
\textbf{Descripción} & Verifica que el campo PublisherDate en Books y el campo review\_time en Ratings respeten el formato correcto de fecha dd/mm/yyyy \\
\hline
\textbf{Aplicado a} & Atributo ⟪PublisherDate⟫ en la tabla NL\_Books y atributo ⟪review\_time⟫ en la tabla NL\_Ratings \\
\hline
\end{tabular}
}
\end{center}


% --------------------------------------------



\subsubsection{duplicated\_authors}

\begin{center}
\resizebox{\textwidth}{!}{
\begin{tabular}{|p{5cm}|p{10cm}|}
\hline
\multicolumn{2}{|c|}{\cellcolor{blue!30} \textbf{MÉTRICA}} \\
\hline
\textbf{Nombre} & duplicated\_authors \\
\hline
\textbf{Descripción} & Indica cuántas tuplas nombran más de un autor \\
\hline
\textbf{Granularidad} & Columna \\
\hline
\textbf{Dominio del Resultado} & [0...1] \\
\hline
\end{tabular}
}
\end{center}

\begin{center}
\resizebox{\textwidth}{!}{
\begin{tabular}{|p{5cm}|p{10cm}|}
\hline
\multicolumn{2}{|c|}{\cellcolor{blue!30} \textbf{MÉTODO}} \\
\hline
\textbf{Descripción} & Implementa la métrica duplicated\_authors para obtener el porcentaje de tuplas que indican más de un autor \\
\hline
\textbf{Tipos de datos de entrada} & String \\
\hline
\textbf{Tipos de datos de salida} & float \\
\hline
\textbf{Algoritmo} & \texttt{DEFINE @tuplasDuplicadas = SELECT COUNT(*) FROM NL\_Authors WHERE AuthorID LIKE '\%,\%'} \newline
\texttt{DEFINE @tuplasTotales = SELECT COUNT(*) FROM NL\_Authors} \newline
\texttt{SELECT @tuplasDuplicadas / @tuplasTotales} \\
\hline
\end{tabular}
}
\end{center}

\begin{center}
\resizebox{\textwidth}{!}{
\begin{tabular}{|p{5cm}|p{10cm}|}
\hline
\multicolumn{2}{|c|}{\cellcolor{blue!30} \textbf{MÉTODO APLICADO}} \\
\hline
\textbf{Tipo} & Medición \\
\hline
\textbf{Descripción} & Divide las tuplas que indican más de un autor sobre las tuplas totales \\
\hline
\textbf{Aplicado a} & Atributo ⟪AuthorID⟫ en la tabla NL\_Books \\
\hline
\end{tabular}
}
\end{center}









% \clearpage







% ---------------------
% % el excel que falta
% ELIMINAR PAGINA. ACÁ VA EL PDF DEL MODELO DE CALIDAD.

% \clearpage

% ELIMINAR PAGINA. ACÁ VA EL PDF DEL MODELO DE CALIDAD.

% \clearpage

% ELIMINAR PAGINA. ACÁ VA EL PDF DEL MODELO DE CALIDAD.

% \clearpage

% ELIMINAR PAGINA. ACÁ VA EL PDF DEL MODELO DE CALIDAD.

% \clearpage






 


\newgeometry{top=2.5cm, bottom=1.5cm, left=1.5cm, right=2.5cm}
% % \afterpage{%
% \begin{landscape}
% \subsection{Modelo de calidad de datos contextual}
% \label{ST4_modelo_calidad_contextual}

% \begin{figure}[H]
%     \centering
%     \noindent
%     \includegraphics[width=1.6\textheight]{Fase 2/Modelo de calidad (entrega 3) (1).pdf}
%     \caption*{}
%     \label{fig:ModeloCalidad1}
% \end{figure}

% \begin{figure}[H]
%     \centering
%     \noindent
%     \includegraphics[width=1.6\textheight]{Fase 2/Modelo de calidad (entrega 3) (2).pdf}
%     \caption*{}
%     \label{fig:ModeloCalidad2}
% \end{figure}

% \begin{figure}[H]
%     \centering
%     \noindent
%     \includegraphics[width=1.6\textheight]{Fase 2/Modelo de calidad (entrega 3) (3).pdf}
%     \caption*{}
%     \label{fig:ModeloCalidad3}
% \end{figure}

% \begin{figure}[H]
%     \centering
%     \noindent
%     \includegraphics[width=1.6\textheight]{Fase 2/Modelo de calidad (entrega 3) (4).pdf}
%     \caption*{}
%     \label{fig:ModeloCalidad4}
% \end{figure}

% % \FloatBarrier
% \end{landscape}
% % }








\begin{landscape}
\subsection{Modelo de calidad de datos contextual}
\label{ST4_modelo_calidad_contextual}

% \begin{center}
  \includegraphics[width=1.5\textheight]{Fase 2/Modelo de calidad (entrega 3) (1).pdf}
% \end{center}


% \begin{center}
  \includegraphics[width=1.5\textheight]{Fase 2/Modelo de calidad (entrega 3) (2).pdf}
% \end{center}


% \begin{center}
  \includegraphics[width=1.5\textheight]{Fase 2/Modelo de calidad (entrega 3) (3).pdf}
  % \captionof{figure}{Modelo de calidad - Parte 3}
% \end{center}


% \begin{center}
  \includegraphics[width=1.5\textheight]{Fase 2/Modelo de calidad (entrega 3) (4).pdf}
% \end{center}

\end{landscape}
\restoregeometry
\input{Fase 2/fase 2 - ST5 y ST6}

\section{Entradas y salidas de la Fase 2 completa}
Como preparación para abordar la Fase 3 del modelo CaDQM, se presentan a continuación las entradas y salidas generales de la Fase 2.

\begin{center}
    \resizebox{\textwidth}{!}{ % Ajusta la tabla al ancho de la página
    \begin{tabular}{| p{8cm} | p{8cm} |} % Columnas con ancho ajustado
        \hline
        \multicolumn{2}{|c|}{\cellcolor{blue!30} \textbf{Entradas y salidas}} \\
        \hline
        \rowcolor{blue!15}
        \textbf{Entradas} & \textbf{Salidas} \\
        \hline
        Reporte del análisis de requerimientos de usuarios (\ref{ST3_reqUsuarios})& Reporte de problemas de CD priorizados (\ref{ST4_problemas_priorizados})\\
        \hline
        Reporte del análisis de datos (\ref{ST2_SubsectionAnalisis}) & Modelo de CD (\ref{ST4_modelo_calidad_contextual})\\
        \hline
        Reporte de problemas de CD (\autoref{Problemas_ST2}) &  Especificación de la BD de metadatos de CD (\ref{Diseño_BDQ})\\
        \hline
        Modelo de Contexto (\ref{ST3_SubsectionContexto}) & Reporte de medición de la CD (\ref{ImplementacionMetodos})\\
        \hline
         & Modelo de contexto (\ref{Contexto_ST5})\\
        \hline
         & Reporte de evaluación de CD (\ref{ImplementacionMetodos})\\
        \hline
    \end{tabular}
    }
\end{center}

\noindent CD: Calidad de datos.\\
BD: Base de datos.





\section{Descripción del desarrollo del trabajo}

% forma de trabajo, herramientas usadas, desafíos enfrentados, etc.
% resultados obtenidos en cada una de las actividades

Siguiendo la metodología CaQDM en la fase 2 el primer paso fue repasar los problemas de calidad analizados en la fase 1 y poder identificar potenciales nuevos problemas. 

 Una vez identificados, procedimos a asignar una prioridad a cada uno tomando en cuenta los requerimientos de calidad, las reglas de negocio, los requerimientos del sistema y las necesidades de filtrado del cliente.

Luego definimos la estructura de la base de datos que usaríamos para almacenar las distintas dimensiones, factores, métricas, métodos, métodos aplicados y ejecuciones de las distintas metricas. 

Una vez pronta la estructura de la base, procedimos a insertar todos los datos y a crear distintos procedimientos almacenados donde se encuentra la lógica de distintos algoritmos de los métodos aplicados. 
Con la estructura pronta y los datos cargados solo restaba ejecutar los distintos métodos aplicados y almacenar los resultados.

 Finalmente armamos el modelo de calidad con sus distintas métricas y métodos tomando en cuenta las prioridades de los problemas de calidad.

\section{Conclusiones de la fase 2}

 La implementación de la segunda fase del modelo CaDQM en este proyecto permitió establecer un enfoque estructurado para diagnosticar el estado actual de la calidad de los datos, considerando los problemas detectados y la prioridad asignada a cada uno.

Mediante las distintas etapas logramos cuantificar la calidad de los datos mediante métricas especificas obtenidas analizando la realidad del problema. Luego aplicando dichas métricas y manejando distintos umbrales pudimos transformar estas métricas en distintas valoraciones de calidad importantes para ayudarnos a entender el estado de calidad de los datos.

 Finalmente, logramos entender el origen de los posibles problemas de calidad que pueden presentar un conjunto de datos y, tomando en cuenta las distintas métricas métodos implementados junto a los umbrales que elegimos, pudimos obtener procedimientos que nos ayudarán a entender estado de la calidad de los datos estudiados de una forma fácil de entender y comunicar.
\chapter{Fase 3 - Data Quality Improvement}

\section{Entradas y salidas de la Fase 3 completa}
Dado que la Fase 3 del modelo CaDQM no se ejecutará completamente y siguiendo cada una de sus etapas, sino que se hará de manera resumida, se presentan a continuación las entradas y salidas esperadas de cada una de las etapas de la fase:


\begin{center}
    \resizebox{\textwidth}{!}{ % Ajusta la tabla al ancho de la página
    \begin{tabular}{| p{8cm} | p{8cm} |} % Columnas con ancho ajustado
        \hline
        \multicolumn{2}{|c|}{\cellcolor{blue!30} \textbf{Entradas y salidas de la ST7}} \\
        \hline
        \rowcolor{blue!15}
        \textbf{Entradas} & \textbf{Salidas} \\
        \hline
        Reporte de evaluación de CD (\ref{ImplementacionMetodos}) & Reporte con los problemas de CD seleccionados y priorizados de acuerdo a sus causas (\ref{Fase3_causas})\\
        \hline
    \end{tabular}
    }
\end{center}



\begin{center}
    \resizebox{\textwidth}{!}{ % Ajusta la tabla al ancho de la página
    \begin{tabular}{| p{8cm} | p{8cm} |} % Columnas con ancho ajustado
        \hline
        \multicolumn{2}{|c|}{\cellcolor{blue!30} \textbf{Entradas y salidas de la ST8}} \\
        \hline
        \rowcolor{blue!15}
        \textbf{Entradas} & \textbf{Salidas} \\
        \hline
        Reporte con los problemas de CD seleccionados y priorizados de acuerdo a sus causas (\ref{Fase3_causas}) & Reporte de análisis de costos\\
        \hline
         & Plan de mejora de la CD (\ref{fase3_plan_de_mejora})\\
        \hline
    \end{tabular}
    }
\end{center}




\begin{center}
    \resizebox{\textwidth}{!}{ % Ajusta la tabla al ancho de la página
    \begin{tabular}{| p{8cm} | p{8cm} |} % Columnas con ancho ajustado
        \hline
        \multicolumn{2}{|c|}{\cellcolor{blue!30} \textbf{Entradas y salidas de la ST9}} \\
        \hline
        \rowcolor{blue!15}
        \textbf{Entradas} & \textbf{Salidas} \\
        \hline
        Reporte de análisis de costos & Reporte de ejecución del plan de mejora de CD\\
        \hline
        Plan de mejora de la CD (\ref{fase3_plan_de_mejora}) & Data at hand mejorados\\
        \hline
    \end{tabular}
    }
\end{center}

\noindent CD: Calidad de datos.\\
BD: Base de datos.



\section{Análisis de causas}
\label{Fase3_causas}

Los datos disponibles en el \textit{data at hand} provienen de dos bases de datos distintas, cada una con sus propios atributos, rangos de valores y formatos definidos. Estas diferencias provocan diversas inconsistencias en la base unificada, vinculadas a los problemas de calidad P1, P2, P8 y P10. Además, como consecuencia directa de la incompatibilidad entre los atributos, se genera una gran cantidad de valores nulos, lo que da lugar al problema P11. Estos errores pueden considerarse de prioridad media, ya que, si bien no impiden directamente el uso de los datos, sí afectan su integridad y consistencia general.

Por otro lado, ambas tablas ya presentaban errores antes de su unificación, producto de un diseño deficiente y de la aparente falta de restricciones sobre los datos ingresados. Esta causa está asociada a los problemas P6, P9, P12, P13, P14 y P15. Dado que estas fallas impactan directamente en la estructura y confiabilidad del sistema, se consideran de prioridad alta.

Finalmente, una fuente adicional de errores es la presencia de errores de tipeo, derivados del ingreso manual de datos. Estos se reflejan en los problemas P3, P4, P5 y P7, y se consideran de prioridad baja.





\section{Plan de mejora}
\label{fase3_plan_de_mejora}

Siguiendo el principio de que toda reingeniería o modificación de procesos debe orientarse a corregir y prevenir errores, procurando mantener la mayor calidad de datos posible a futuro, la mejora principal propuesta consiste en una reestructuración de la base de datos. La nueva estructura se presenta en la \autoref{NewBaseNL}.

Esta nueva base fue diseñada con el objetivo de normalizar formatos y nombres, evitar redundancias y reducir al mínimo el ingreso manual de datos por parte del usuario, con el fin de mitigar errores de tipeo. 

Además, se incorporan restricciones de integridad y de dominio que permiten controlar la validez de los datos ingresados, mejorando la consistencia general del sistema.

\begin{figure}[htbp]
    \centering
    \makebox[\linewidth][c]{%
        \includegraphics[width=0.8\linewidth]{Fase 3/Nueva base propuesta.pdf}
    }
    \caption{Estructura de la nueva base de datos propuesta}
    \label{NewBaseNL}
\end{figure}

Para reducir aún más el ingreso manual, se propone el uso de diccionarios o listas de selección para campos como ciudades, géneros, autores y editoriales, permitiendo la introducción manual únicamente en los casos en que los valores no se encuentren previamente cargados.

Respecto a los datos ya existentes que presentan errores o formatos diferentes al estándar (por ejemplo, nombres en minúscula), se recomienda implementar funciones automáticas de corrección y estandarización. Para aquellos registros con valores nulos o errores que no puedan corregirse automáticamente, será necesaria una revisión y corrección manual.

Finalmente, se sugiere capacitar al personal encargado del ingreso y mantenimiento de los datos, a fin de garantizar el uso correcto del sistema. Asimismo, se recomienda establecer controles periódicos de calidad de datos, que permitan monitorizar el estado general de la base y tomar acciones correctivas cuando sea necesario.





\section{Descripción del desarrollo del trabajo propuesto}

Si bien no se implementó toda la Fase 3, se entiende que su desarrollo podría haberse llevado a cabo en conjunto con el cliente. 
A continuación se propone un posible enfoque para su implementación:

\begin{itemize}
    \item \textbf{Validación del plan de mejora:} confirmar que las acciones propuestas están alineadas con los requerimientos de calidad definidos previamente y con los intereses del cliente.
    
    \item \textbf{Estimación de recursos:} calcular tiempo y costo necesarios para llevar adelante el plan de mejora evaluando  su viabilidad.
    
    \item \textbf{Alineación con el cliente:} mantener una comunicación con el cliente para asegurar que el plan validado responda sus necesidades.
    
    \item \textbf{Aplicación de mejoras :} ejecutar los distintos algoritmos propuestos (3.2.3). Para cada uno se evaluara el impacto sobre la calidad de los datos utilizando las métricas definidas en la Fase 2 como mecanismo de comparación y validación.
\end{itemize}


\section{Conclusiones de la fase 3}

La fase 3 es el cierre del ciclo propuesto por CaDQM y si bien no se implementó de forma completa pudimos entender la importancia del mismo y sobre todo el porque de muchos elementos propuestos en fases  previas. 

En este caso entendemos que es una etapa en la que es crucial la comunicación fluida con el cliente (si bien todas las etapas, en cierta forma, lo son) ya que es donde se cierra el proyecto y donde se dejan ver los resultados y quizá esto complica su implementación.

Nos hubiera gustado poder atacar algunos de los puntos mencionados en esta fase (como la estimación o la aplicación de las soluciones propuestas) aunque logramos dejar en claro la forma en que lo haríamos.



\chapter{Conclusiones finales}

Este proyecto nos permitió implementar las distintas fases de CaDQM, desde la planificación inicial hasta la propuesta de acciones de mejora (de forma parcial). 

Durante el proceso aplicamos las distintas técnicas y herramientas propuestas para identificar, medir y evaluar problemas de calidad de datos sobre una base integrada proveniente de dos fuentes distintas, simulando un caso real.

Cada una de las fases nos permitió entender distintos problemas sobre la calidad de los datos y como cada etapa se alimentaba de la anterior pudimos ver como es importante tener una base firme para llegar a las etapas mas avanzadas con un proyecto solido (y en concordancia con el cliente).

Si bien no se implemento de forma total la Fase 3, pudimos elaborar un plan de mejora concreto y proponer una estrategia para su desarrollo en conjunto con el cliente.

En resumen valoramos el poder aplicar una metodología completa sobre un caso de un cliente real, con muchas dimensiones y problemas a considerar así como distintas decisiones a tomar. Esta experiencia fue útil para poder bajar a tierra los distintos temas vistos en clase y ademas poder tener criterio propio a la hora de enfrentarnos con posibles problemas de calidad de datos en la vida profesional.

Finalmente el trabajo nos permitió comprender y aplicar el modelo CaDQM en profundidad, destacando tanto sus ventajas como los distintos problemas o desafíos que implica llevarlo a la práctica.

\vspace{0.3cm}

\noindent Como complemento al informe dejamos adjuntos una serie de scripts que incluyen la base de datos de metadatos de calidad y todos los algoritmos implementados para la medición y análisis realizados durante el proyecto.

\end{document}