\chapter{Fase 2 - Data Quality Assessment}

\section{ST4: DQ Model Definition}

\subsection{Entradas y salidas}
\begin{center}
    \resizebox{\textwidth}{!}{ % Ajusta la tabla al ancho de la página
    \begin{tabular}{| p{8cm} | p{8cm} |} % Columnas con ancho ajustado
        \hline
        \multicolumn{2}{|c|}{\cellcolor{blue!30} \textbf{Entradas y salidas}} \\
        \hline
        \rowcolor{blue!15}
        \textbf{Entradas} & \textbf{Salidas} \\
        \hline
        Reporte del análisis de requerimientos de usuarios (\ref{ST3_reqUsuarios})& Reporte de problemas de CD priorizados (\ref{ST4_problemas_priorizados})\\
        \hline
        Reporte del análisis de datos (\ref{ST2_SubsectionAnalisis}) & Modelo de CD (\ref{ST4_modelo_calidad_contextual})\\
        \hline
        Reporte de problemas de CD (\autoref{Problemas_ST2}) &  \\
        \hline
        Modelo de Contexto (\ref{ST3_SubsectionContexto}) & \\
        \hline
    \end{tabular}
    }
\end{center}

\noindent CD: Calidad de datos.\\



\subsection{Priorización de los problemas de calidad de datos}
\label{ST4_problemas_priorizados}

La siguiente tabla muestra los problemas de calidad de datos detectados durante la \textit{Fase 1}, ya priorizados.

Para la priorización, se consideró que todos los problemas de calidad asociados a reglas de negocio tuvieran alta prioridad, dado que representan un interés central para el cliente o constituyen requisitos esenciales para el correcto funcionamiento de cualquier sistema de bases de datos.

Los errores de formato o que tienen que ver con una posible futura reestructuración de la base de datos fueron clasificados con una prioridad media, ya que, si bien no impiden el uso de los datos, pueden dificultar su utilización,  presentación o interpretación.

Por otro lado, aquellos problemas relacionados con errores ortográficos, tipográficos o inconsistencias en la escritura se asignaron con prioridad baja. La única excepción a esta regla es el caso del \textit{P3} con prioridad media, ya que un error en el \textit{Publisher} interfiere directamente con la necesidad de filtrado \textit{F2}, imposibilitando parte de las tareas del usuario \textit{U1}.


\begin{center}
\resizebox{\textwidth}{!}{
\renewcommand{\arraystretch}{1.3} % Aumenta el espacio entre filas
\begin{tabular}{|p{5cm}|p{8.5cm}|p{2cm}|}
    \hline
    \multicolumn{3}{|c|}{\cellcolor{blue!30} \textbf{Problemas detectados en la Calidad de los Datos}} \\
    \hline
    \rowcolor{blue!15}
    \textbf{Campo} & \textbf{Problema de calidad} & \textbf{Prioridad} \\
    \hline
    NL\_Books.ISBN & P1: Entradas no respetan el formato ISBN ya que algunos están en formato ASID. & Alta \\
    \hline
    NL\_Books.PublisherDate & P2: Las fechas tienen distinto formato. & Media \\
    \hline
    NL\_Books.PublisherID & P3: Mismo publisher escrito de forma distinta. & Media \\
    \hline
    NL\_Books.Title & P4: Títulos mal escritos. & Baja \\
    \hline
    NL\_Books.AuthorID & P5: Mismo autor escrito de forma distinta. & Baja \\
    \hline
    NL\_USERS.Age & P6: Valores de edad poco coherentes (por ejemplo 0). & Alta \\
    \hline
    NL\_USERS.Location & P7: Ciudades mal escritas. & Baja \\
    \hline
    NL\_RATINGS.review\_time & P8: Formato de fecha/hora inconsistente. & Media \\
    \hline
    NL\_RATINGS.review\_score & P9: Valores no numéricos en la puntuación. & Media \\
    \hline
    NL\_RATINGS.review\_score & P10: Los valores importados entre L1 y L2 manejan distintas escalas (L1 puntúa de 0 a 5 y L2 de 0 a 10). & Media \\
    \hline
    Base de datos & P11: Gran cantidad de nulos en muchos de los atributos, incluyendo claves. & Alta \\
    \hline
    NL\_Books & P12: No hay campo de rating promedio del libro (podría calcularse). & Media
    \\
    \hline
    NL\_RATINGS.Helpfullness & P13: Este atributo en realidad debería ser dos atributos diferentes: cantidad de votaciones en esa review y cantidad de votaciones que consideraron útil esa review. & Media
    \\
    \hline
    NL\_Books.AuthorID & P14: Libros indican autores de forma distinta cuando tienen mas de uno (dos libros con autores \{A,B\} y \{B,A\} deben considerarse con los mismos autores). & Baja 
    \\
    \hline
     NL\_RATINGS.review\_score & P15: Podría haber valores fueron del rango 0 a 10. &  Media
    \\
    \hline
\end{tabular}
}
\end{center}







\subsection{Métricas}

\subsubsection{check\_ISBN}

\begin{center}
\resizebox{\textwidth}{!}{
\begin{tabular}{| p{3.5cm} | p{11.5cm} |}
    \hline
    \multicolumn{2}{|c|}{\cellcolor{blue!30} \textbf{Métrica}} \\
    \hline
    \textbf{Nombre} & \textbf{check\_ISBN} \\
    \hline
    \textbf{Descripción} & Controla si el valor es un ISBN válido. \\
    \hline
    \textbf{Granularidad} & Celda \\
    \hline
    \textbf{Dominio del Resultado} & \{0, 1\} \\
    \hline
\end{tabular}
}
\end{center}

% \vspace{1em}

\begin{center}
\resizebox{\textwidth}{!}{
\begin{tabular}{| p{3.5cm} | p{11.5cm} |}
    \hline
    \multicolumn{2}{|c|}{\cellcolor{blue!30} \textbf{Método}} \\
    \hline
    \textbf{Descripción} & Implementa la métrica Check\_ISBN teniendo en cuenta la estructura de un código ISBN. \\
    \hline
    \textbf{Tipos de datos de entrada} & String \\
    \hline
    \textbf{Tipos de datos de salida} & Boolean \\
    \hline
    \textbf{Algoritmo} & 
\texttt{def Check\_ISBN(codigo):} \\ 
& \texttt{\ \ size = len(codigo)} \\
& \texttt{\ \ \# Chequea para el caso de un ISBN10} \\
& \texttt{\ \ if (size == 10):} \\
& \texttt{\ \ \ \ verifica que los primeros 9 caracteres sean dígitos} \\
& \texttt{\ \ \ \ verifica que el último caracter sea un dígito o una "X"} \\
& \texttt{\ \ \ \ verifica que el dígito verificador sea correcto} \\
& \texttt{\ \ \# Chequea para el caso de un ISBN13} \\
& \texttt{\ \ if (size == 13):} \\
& \texttt{\ \ \ \ verifica que todos los caracteres sean dígitos} \\
& \texttt{\ \ \ \ verifica que el dígito verificador sea correcto} \\
    \hline
\end{tabular}
}
\end{center}

% \vspace{1em}

\begin{center}
\resizebox{\textwidth}{!}{
\begin{tabular}{| p{3.5cm} | p{11.5cm} |}
    \hline
    \multicolumn{2}{|c|}{\cellcolor{blue!30} \textbf{Método aplicado}} \\
    \hline
    \textbf{Tipo} & Medición \\
    \hline
    \textbf{Descripción} & Utiliza el algoritmo de cálculo de ISBN para verificar la validez del dato. \\
    \hline
    \textbf{Aplicado a} & Atributo ⟪ISBN⟫ de la tabla \texttt{NL\_Books} \\
    \hline
\end{tabular}
}
\end{center}




% -------------

\subsubsection{check\_edades}

\begin{center}
\resizebox{\textwidth}{!}{
\begin{tabular}{|p{3.5cm} | p{11.5cm}|}
\hline
\multicolumn{2}{|c|}{\cellcolor{blue!30} \textbf{Métrica}} \\
\hline
\textbf{Nombre} & check\_edades \\
\hline
\textbf{Descripción} & Controla si la edad es válida. \\
\hline
\textbf{Granularidad} & Celda \\
\hline
\textbf{Dominio del Resultado} & \{0, 1\} \\
\hline
\end{tabular}
}
\end{center}


\begin{center}
\resizebox{\textwidth}{!}{
\begin{tabular}{|p{3.5cm} | p{11.5cm}|}
\hline
\multicolumn{2}{|c|}{\cellcolor{blue!30} \textbf{Método}} \\
\hline
\textbf{Descripción} & Implementa la métrica check\_edades verificando si la edad tiene formato correcto y es un valor razonable. \\
\hline
\textbf{Tipos de datos de entrada} & String \\
\hline
\textbf{Tipos de datos de salida} & Boolean \\
\hline
\textbf{Algoritmo} & 
\texttt{IsNumeric(edad) AND edad > 0 AND edad < 100} \\
\hline
\end{tabular}
}
\end{center}



\begin{center}
\resizebox{\textwidth}{!}{
\begin{tabular}{|p{3.5cm} | p{11.5cm}|}
\hline
\multicolumn{2}{|c|}{\cellcolor{blue!30} \textbf{Método aplicado}} \\
\hline
\textbf{Tipo} & Medición \\
\hline
\textbf{Descripción} & Verifica que tenga un valor numérico entre 1 y 99. \\
\hline
\textbf{Aplicado a} & Atributo ⟪Age⟫ de la tabla \texttt{NL\_Users} \\
\hline
\end{tabular}
}
\end{center}


% -------------------------

\subsubsection{check\_price}

\begin{center}
\resizebox{\textwidth}{!}{
\begin{tabular}{|p{3.5cm} | p{11.5cm}|}
\hline
\multicolumn{2}{|c|}{\cellcolor{blue!30} \textbf{Métrica}} \\
\hline
\textbf{Nombre} & check\_price \\
\hline
\textbf{Descripción} & Controla si el precio es válido. \\
\hline
\textbf{Granularidad} & Celda \\
\hline
\textbf{Dominio del Resultado} & \{0, 1\} \\
\hline
\end{tabular}
}
\end{center}


\begin{center}
\resizebox{\textwidth}{!}{
\begin{tabular}{|p{3.5cm} | p{11.5cm}|}
\hline
\multicolumn{2}{|c|}{\cellcolor{blue!30} \textbf{Método}} \\
\hline
\textbf{Descripción} & Implementa la métrica check\_price verificando si el valor tiene formato correcto. \\
\hline
\textbf{Tipos de datos de entrada} & String \\
\hline
\textbf{Tipos de datos de salida} & Boolean \\
\hline
\textbf{Algoritmo} & \texttt{IsNumeric(price) AND price >= 0} \\
\hline
\end{tabular}
}
\end{center}


\begin{center}
\resizebox{\textwidth}{!}{
\begin{tabular}{|p{3.5cm} | p{11.5cm}|}
\hline
\multicolumn{2}{|c|}{\cellcolor{blue!30} \textbf{Método aplicado}} \\
\hline
\textbf{Tipo} & Medición \\
\hline
\textbf{Descripción} & Verifica que tenga un valor numérico positivo con una consulta SQL. \\
\hline
\textbf{Aplicado a} & Atributo ⟪Price⟫ de la tabla \texttt{NL\_Books} \\
\hline
\end{tabular}
}
\end{center}


% ---------------------------



\subsubsection{duplicate\_ratio}

\begin{center}
\resizebox{\textwidth}{!}{
\begin{tabular}{|p{3.5cm} | p{11.5cm}|}
\hline
\multicolumn{2}{|c|}{\cellcolor{blue!30} \textbf{Métrica}} \\
\hline
\textbf{Nombre} & duplicate\_ratio \\
\hline
\textbf{Descripción} & Da el porcentaje de valores duplicados. \\
\hline
\textbf{Granularidad} & Columna \\
\hline
\textbf{Dominio del Resultado} & [0...1] \\
\hline
\end{tabular}
}
\end{center}


\begin{center}
\resizebox{\textwidth}{!}{
\begin{tabular}{|p{3.5cm} | p{11.5cm}|}
\hline
\multicolumn{2}{|c|}{\cellcolor{blue!30} \textbf{Método}} \\
\hline
\textbf{Descripción} & Implementa la métrica duplicate\_ratio sobre un atributo dado. \\
\hline
\textbf{Tipos de datos de entrada} & String \\
\hline
\textbf{Tipos de datos de salida} & Float \\
\hline
\textbf{Algoritmo} & 
\texttt{def duplicate\_ratio(data, column):} \\ 
& \texttt{\ \ \ \ total = len(data)} \\
& \texttt{\ \ \ \ duplicated = data.duplicated(subset=[column]).sum()} \\
& \texttt{\ \ \ \ return duplicated / total} \\
\hline
\end{tabular}
}
\end{center}


\begin{center}
\resizebox{\textwidth}{!}{
\begin{tabular}{|p{3.5cm} | p{11.5cm}|}
\hline
\multicolumn{2}{|c|}{\cellcolor{blue!30} \textbf{Método aplicado}} \\
\hline
\textbf{Tipo} & Medición \\
\hline
\textbf{Descripción} & Dados los datos y el atributo, calcula el porcentaje de valores duplicados en ese atributo. \\
\hline
\textbf{Aplicado a} & Atributos ⟪ISBN⟫, ⟪AuthorID⟫, ⟪PublisherID⟫, ⟪User\_ID⟫ de la tabla \texttt{NL\_Books}. \newline
Atributo ⟪ID⟫ de la tabla \texttt{NL\_Users}. \\
\hline
\end{tabular}
}
\end{center}


% ---------------------------------------



\subsubsection{check\_RN1}

\begin{center}
\resizebox{\textwidth}{!}{
\begin{tabular}{|p{3.5cm} | p{11.5cm}|}
\hline
\multicolumn{2}{|c|}{\cellcolor{blue!30} \textbf{Métrica}} \\
\hline
\textbf{Nombre} & check\_RN1 \\
\hline
\textbf{Descripción} & Da el porcentaje de entradas de la tabla que tienen al menos un campo vacío entre los atributos \texttt{isbn}, \texttt{titulo}, \texttt{autor} y \texttt{editor}. \\
\hline
\textbf{Granularidad} & Conjunto de columnas \\
\hline
\textbf{Dominio del Resultado} & [0...1] \\
\hline
\end{tabular}
}
\end{center}


\begin{center}
\resizebox{\textwidth}{!}{
\begin{tabular}{|p{3.5cm} | p{11.5cm}|}
\hline
\multicolumn{2}{|c|}{\cellcolor{blue!30} \textbf{Método}} \\
\hline
\textbf{Descripción} & Implementa la métrica check\_RN1 sobre una tabla dada. \\
\hline
\textbf{Tipos de datos de entrada} & String \\
\hline
\textbf{Tipos de datos de salida} & Float \\
\hline
\textbf{Algoritmo} & 
\texttt{def Check\_RN1(datos):} \\
& \texttt{\ \ \ \ \# Seleccionar solo las columnas relevantes (isbn, titulo, autor, editor)} \\
& \texttt{\ \ \ \ datos\_relevantes = datos[['isbn', 'titulo', 'autor', 'editor']]} \\
& \texttt{\ \ \ \ \# Verificar cuántas filas tienen al menos un campo NULL entre estas columnas} \\
& \texttt{\ \ \ \ incompletos = datos\_relevantes.isnull().any(axis=1).sum()} \\
& \texttt{\ \ \ \ \# Calcular la proporción de filas incompletas} \\
& \texttt{\ \ \ \ return incompletos / len(datos)} \\
\hline
\end{tabular}
}
\end{center}


\begin{center}
\resizebox{\textwidth}{!}{
\begin{tabular}{|p{3.5cm} | p{11.5cm}|}
\hline
\multicolumn{2}{|c|}{\cellcolor{blue!30} \textbf{Método aplicado}} \\
\hline
\textbf{Tipo} & Medición \\
\hline
\textbf{Descripción} & Dada una tabla, calcula el porcentaje de entradas que tienen al menos un campo vacío entre los atributos \texttt{isbn}, \texttt{titulo}, \texttt{autor} y \texttt{editor}. \\
\hline
\textbf{Aplicado a} & Tabla \texttt{NL\_Books}. \\
\hline
\end{tabular}
}
\end{center}



% -----------------------------



\subsubsection{contar\_nulls}

\begin{center}
\resizebox{\textwidth}{!}{
\begin{tabular}{|p{3.5cm} | p{11.5cm}|}
\hline
\multicolumn{2}{|c|}{\cellcolor{blue!30} \textbf{Métrica}} \\
\hline
\textbf{Nombre} & contar\_nulls \\
\hline
\textbf{Descripción} & Calcula el porcentaje de entradas vacías en una columna. \\
\hline
\textbf{Granularidad} & Columna \\
\hline
\textbf{Dominio del Resultado} & [0...1] \\
\hline
\end{tabular}
}
\end{center}

\begin{center}
\resizebox{\textwidth}{!}{
\begin{tabular}{|p{3.5cm} | p{11.5cm}|}
\hline
\multicolumn{2}{|c|}{\cellcolor{blue!30} \textbf{Método}} \\
\hline
\textbf{Descripción} & Implementa la métrica contar\_nulls. \\
\hline
\textbf{Tipos de datos de entrada} & String \\
\hline
\textbf{Tipos de datos de salida} & Float \\
\hline
\textbf{Algoritmo} & \texttt{def porcentaje\_nulos(atributo):} \\
& \texttt{\ \ \ \ nulos = atributo.isnull()} \\
& \texttt{\ \ \ \ contador\_nulos = nulos.sum()} \\
& \texttt{\ \ \ \ porcentaje = contador\_nulos / len(atributo)} \\
& \texttt{\ \ \ \ return porcentaje} \\
\hline
\end{tabular}
}
\end{center}

\begin{center}
\resizebox{\textwidth}{!}{
\begin{tabular}{|p{3.5cm} | p{11.5cm}|}
\hline
\multicolumn{2}{|c|}{\cellcolor{blue!30} \textbf{Método aplicado}} \\
\hline
\textbf{Tipo} & Medición \\
\hline
\textbf{Descripción} & Dado un atributo de una tabla, calcula el porcentaje de entradas vacías. \\
\hline
\textbf{Aplicado a} & Cualquier atributo de cualquier tabla. \\
\hline
\end{tabular}
}
\end{center}


% ---------------------------------------

\subsubsection{concistencia\_ratings}

\begin{center}
\resizebox{\textwidth}{!}{
\begin{tabular}{|p{3.5cm} | p{11.5cm}|}
\hline
\multicolumn{2}{|c|}{\cellcolor{blue!30} \textbf{Métrica}} \\
\hline
\textbf{Nombre} & concistencia\_ratings \\
\hline
\textbf{Descripción} & Verifica que la cantidad de ratings entre las distintas tablas sea coherente. \\
\hline
\textbf{Granularidad} & Conjunto de columnas \\
\hline
\textbf{Dominio del Resultado} & \{0, 1\} \\
\hline
\end{tabular}
}
\end{center}

\begin{center}
\resizebox{\textwidth}{!}{
\begin{tabular}{|p{3.5cm} | p{11.5cm}|}
\hline
\multicolumn{2}{|c|}{\cellcolor{blue!30} \textbf{Método}} \\
\hline
\textbf{Descripción} & Implementa la métrica concistencia\_ratings contando la cantidad de ocurrencias de un ISBN válido en la tabla NL\_ratings y lo compara con lo declarado en el atributo rating\_count de NL\_Books. \\
\hline
\textbf{Tipos de datos de entrada} & String \\
\hline
\textbf{Tipos de datos de salida} & Boolean \\
\hline
\textbf{Algoritmo} & \texttt{def funcion\_consistencia\_ratings(libros, valid\_isbn, ratings):} \\
& \texttt{\ \ \ \ libros\_validos = libros[valid\_isbn==True]} \\
& \texttt{\ \ \ \ for libro in libros\_validos:} \\
& \texttt{\ \ \ \ \ \ \ \ isbn = libro['isbn']} \\
& \texttt{\ \ \ \ \ \ \ \ cantidad = (ratings['isbn'] == isbn).sum()} \\
& \texttt{\ \ \ \ \ \ \ \ contados[isbn] = cantidad} \\
& \texttt{\ \ \ \ rating\_esperado = libros\_validos['rating\_counts']} \\
& \texttt{\ \ \ \ return rating\_esperado == contados} \\
\hline
\end{tabular}
}
\end{center}

\begin{center}
\resizebox{\textwidth}{!}{
\begin{tabular}{|p{3.5cm} | p{11.5cm}|}
\hline
\multicolumn{2}{|c|}{\cellcolor{blue!30} \textbf{Método aplicado}} \\
\hline
\textbf{Tipo} & Agregación \\
\hline
\textbf{Descripción} & Dada una tabla de reviews, una tabla de libros (ambas con el campo isbn) y una lista que indiquen si son válidos o no, determina si hay congruencia entre lo registrado en ambas tablas sobre los ratings. \\
\hline
\textbf{Aplicado a} & Conjunto de atributos (NL\_ratings.ISBN, NL\_books.ISBN, NL\_books.Rating\_Count) \\
\hline
\end{tabular}
}
\end{center}


% ------------------------------------------


\subsubsection{consistencia\_fechas}

\begin{center}
\resizebox{\textwidth}{!}{
\begin{tabular}{|p{3.5cm} | p{11.5cm}|}
\hline
\multicolumn{2}{|c|}{\cellcolor{blue!30} \textbf{Métrica}} \\
\hline
\textbf{Nombre} & consistencia\_fechas \\
\hline
\textbf{Descripción} & Verifica que la fecha de un rating sea posterior a la fecha de publicación del libro. \\
\hline
\textbf{Granularidad} &  Conjunto de columnas \\
\hline
\textbf{Dominio del Resultado} & \{0, 1\} \\
\hline
\end{tabular}
}
\end{center}

\begin{center}
\resizebox{\textwidth}{!}{
\begin{tabular}{|p{3.5cm} | p{11.5cm}|}
\hline
\multicolumn{2}{|c|}{\cellcolor{blue!30} \textbf{Método}} \\
\hline
\textbf{Descripción} & Implementa la métrica concistencia\_fechas comparando las fechas de review\_time de la tabla NL\_reviews y la de PublisherDate de NL\_Books. \\
\hline
\textbf{Tipos de datos de entrada} & String \\
\hline
\textbf{Tipos de datos de salida} & Boolean \\
\hline
\textbf{Algoritmo} & \texttt{def consistencia\_fechas(NL\_reviews, NL\_books):} \\
& \texttt{\ \ \ \ result = []} \\
& \texttt{\ \ \ \ for i in range(len(NL\_reviews)):\ } \\
& \texttt{\ \ \ \ \ \ \ \ isbn = NL\_reviews['isbn'][i]} \\
& \texttt{\ \ \ \ \ \ \ \ fila\_libro = NL\_books[NL\_books['isbn'] == isbn]} \\
& \texttt{\ \ \ \ \ \ \ \ review\_time, PublishedDate = fila\_libro[['review\_time', 'PublishedDate']].values[0]} \\
& \texttt{\ \ \ \ \ \ \ \ result.append(review\_time >= PublishedDate)} \\
& \texttt{\ \ \ \ return result} \\
\hline
\end{tabular}
}
\end{center}

\begin{center}
\resizebox{\textwidth}{!}{
\begin{tabular}{|p{3.5cm} | p{11.5cm}|}
\hline
\multicolumn{2}{|c|}{\cellcolor{blue!30} \textbf{Método aplicado}} \\
\hline
\textbf{Tipo} & Medición \\
\hline
\textbf{Descripción} & Dada una tabla de reviews y una tabla de libros (ambas con el campo isbn), determina si la fecha de la review es posterior a la publicación del libro. \\
\hline
\textbf{Aplicado a} & Conjunto de atributos ⟪NL_Reviews.review_time, NL_Books.PublisherDate⟫  \\
\hline
\end{tabular}
}
\end{center}


% -------------------------------



\subsubsection{missing\_rating\_books}

\begin{center}
\resizebox{\textwidth}{!}{
\begin{tabular}{|p{3.5cm} | p{11.5cm}|}
\hline
\multicolumn{2}{|c|}{\cellcolor{blue!30} \textbf{Métrica}} \\
\hline
\textbf{Nombre} & missing\_rating\_books \\
\hline
\textbf{Descripción} & Para ratings sobre libros, controla la existencia de estos en la base de datos. \\
\hline
\textbf{Granularidad} & Celda \\
\hline
\textbf{Dominio del Resultado} & \{0, 1\} \\
\hline
\end{tabular}
}
\end{center}

\begin{center}
\resizebox{\textwidth}{!}{
\begin{tabular}{|p{3.5cm} | p{11.5cm}|}
\hline
\multicolumn{2}{|c|}{\cellcolor{blue!30} \textbf{Método}} \\
\hline
\textbf{Descripción} & Implementa la métrica check\_rating\_books para ver si los libros indicados en los ratings existen. \\
\hline
\textbf{Tipos de datos de entrada} & String \\
\hline
\textbf{Tipos de datos de salida} & Boolean \\
\hline
\textbf{Algoritmo} & \texttt{SELECT COUNT(*) FROM NL\_Ratings WHERE NOT EXISTS(SELECT * FROM NL\_Books WHERE NL\_Ratings.ISBN = NL\_Books.ISBN)} \\
\hline
\end{tabular}
}
\end{center}

\begin{center}
\resizebox{\textwidth}{!}{
\begin{tabular}{|p{3.5cm} | p{11.5cm}|}
\hline
\multicolumn{2}{|c|}{\cellcolor{blue!30} \textbf{Método aplicado}} \\
\hline
\textbf{Tipo} & Medición \\
\hline
\textbf{Descripción} & Verifica que el libro en NL\_Ratings exista en NL\_Books \\
\hline
\textbf{Aplicado a} & Atributos ⟪ISBN⟫ en las tablas NL\_Books y NL\_Ratings \\
\hline
\end{tabular}
}
\end{center}


% ------------------------------------

\subsubsection{date\_format}

\begin{center}
\resizebox{\textwidth}{!}{
\begin{tabular}{|p{3.5cm} | p{11.5cm}|}
\hline
\multicolumn{2}{|c|}{\cellcolor{blue!30} \textbf{Métrica}} \\
\hline
\textbf{Nombre} & date\_format \\
\hline
\textbf{Descripción} & Controla el formato de fecha de todas las columnas tipo fecha \\
\hline
\textbf{Granularidad} & Celda \\
\hline
\textbf{Dominio del Resultado} & \{0, 1\} \\
\hline
\end{tabular}
}
\end{center}

\begin{center}
\resizebox{\textwidth}{!}{
\begin{tabular}{|p{3.5cm} | p{11.5cm}|}
\hline
\multicolumn{2}{|c|}{\cellcolor{blue!30} \textbf{Método}} \\
\hline
\textbf{Descripción} & Implementa la métrica date\_format para controlar que el formato de fecha sea correcto \\
\hline
\textbf{Tipos de datos de entrada} & String \\
\hline
\textbf{Tipos de datos de salida} & Boolean \\
\hline
\textbf{Algoritmo} & \texttt{TRY\_CONVERT(DATE,date,103)} \\
\hline
\end{tabular}
}
\end{center}

\begin{center}
\resizebox{\textwidth}{!}{
\begin{tabular}{|p{3.5cm} | p{11.5cm}|}
\hline
\multicolumn{2}{|c|}{\cellcolor{blue!30} \textbf{Método aplicado}} \\
\hline
\textbf{Tipo} & Medición \\
\hline
\textbf{Descripción} & Verifica que el campo PublisherDate en Books y el campo review\_time en Ratings respeten el formato correcto de fecha dd/mm/yyyy \\
\hline
\textbf{Aplicado a} & Atributo ⟪PublisherDate⟫ en la tabla NL\_Books y atributo ⟪review\_time⟫ en la tabla NL\_Ratings \\
\hline
\end{tabular}
}
\end{center}


% --------------------------------------------



\subsubsection{duplicated\_authors}

\begin{center}
\resizebox{\textwidth}{!}{
\begin{tabular}{|p{5cm}|p{10cm}|}
\hline
\multicolumn{2}{|c|}{\cellcolor{blue!30} \textbf{MÉTRICA}} \\
\hline
\textbf{Nombre} & duplicated\_authors \\
\hline
\textbf{Descripción} & Indica cuántas tuplas nombran más de un autor \\
\hline
\textbf{Granularidad} & Columna \\
\hline
\textbf{Dominio del Resultado} & [0...1] \\
\hline
\end{tabular}
}
\end{center}

\begin{center}
\resizebox{\textwidth}{!}{
\begin{tabular}{|p{5cm}|p{10cm}|}
\hline
\multicolumn{2}{|c|}{\cellcolor{blue!30} \textbf{MÉTODO}} \\
\hline
\textbf{Descripción} & Implementa la métrica duplicated\_authors para obtener el porcentaje de tuplas que indican más de un autor \\
\hline
\textbf{Tipos de datos de entrada} & String \\
\hline
\textbf{Tipos de datos de salida} & float \\
\hline
\textbf{Algoritmo} & \texttt{DEFINE @tuplasDuplicadas = SELECT COUNT(*) FROM NL\_Authors WHERE AuthorID LIKE '\%,\%'} \newline
\texttt{DEFINE @tuplasTotales = SELECT COUNT(*) FROM NL\_Authors} \newline
\texttt{SELECT @tuplasDuplicadas / @tuplasTotales} \\
\hline
\end{tabular}
}
\end{center}

\begin{center}
\resizebox{\textwidth}{!}{
\begin{tabular}{|p{5cm}|p{10cm}|}
\hline
\multicolumn{2}{|c|}{\cellcolor{blue!30} \textbf{MÉTODO APLICADO}} \\
\hline
\textbf{Tipo} & Medición \\
\hline
\textbf{Descripción} & Divide las tuplas que indican más de un autor sobre las tuplas totales \\
\hline
\textbf{Aplicado a} & Atributo ⟪AuthorID⟫ en la tabla NL\_Books \\
\hline
\end{tabular}
}
\end{center}









% \clearpage







% ---------------------
% % el excel que falta
% ELIMINAR PAGINA. ACÁ VA EL PDF DEL MODELO DE CALIDAD.

% \clearpage

% ELIMINAR PAGINA. ACÁ VA EL PDF DEL MODELO DE CALIDAD.

% \clearpage

% ELIMINAR PAGINA. ACÁ VA EL PDF DEL MODELO DE CALIDAD.

% \clearpage

% ELIMINAR PAGINA. ACÁ VA EL PDF DEL MODELO DE CALIDAD.

% \clearpage






 


\newgeometry{top=2.5cm, bottom=1.5cm, left=1.5cm, right=2.5cm}
% % \afterpage{%
% \begin{landscape}
% \subsection{Modelo de calidad de datos contextual}
% \label{ST4_modelo_calidad_contextual}

% \begin{figure}[H]
%     \centering
%     \noindent
%     \includegraphics[width=1.6\textheight]{Fase 2/Modelo de calidad (entrega 3) (1).pdf}
%     \caption*{}
%     \label{fig:ModeloCalidad1}
% \end{figure}

% \begin{figure}[H]
%     \centering
%     \noindent
%     \includegraphics[width=1.6\textheight]{Fase 2/Modelo de calidad (entrega 3) (2).pdf}
%     \caption*{}
%     \label{fig:ModeloCalidad2}
% \end{figure}

% \begin{figure}[H]
%     \centering
%     \noindent
%     \includegraphics[width=1.6\textheight]{Fase 2/Modelo de calidad (entrega 3) (3).pdf}
%     \caption*{}
%     \label{fig:ModeloCalidad3}
% \end{figure}

% \begin{figure}[H]
%     \centering
%     \noindent
%     \includegraphics[width=1.6\textheight]{Fase 2/Modelo de calidad (entrega 3) (4).pdf}
%     \caption*{}
%     \label{fig:ModeloCalidad4}
% \end{figure}

% % \FloatBarrier
% \end{landscape}
% % }








\begin{landscape}
\subsection{Modelo de calidad de datos contextual}
\label{ST4_modelo_calidad_contextual}

% \begin{center}
  \includegraphics[width=1.5\textheight]{Fase 2/Modelo de calidad (entrega 3) (1).pdf}
% \end{center}


% \begin{center}
  \includegraphics[width=1.5\textheight]{Fase 2/Modelo de calidad (entrega 3) (2).pdf}
% \end{center}


% \begin{center}
  \includegraphics[width=1.5\textheight]{Fase 2/Modelo de calidad (entrega 3) (3).pdf}
  % \captionof{figure}{Modelo de calidad - Parte 3}
% \end{center}


% \begin{center}
  \includegraphics[width=1.5\textheight]{Fase 2/Modelo de calidad (entrega 3) (4).pdf}
% \end{center}

\end{landscape}
\restoregeometry
\input{Fase 2/fase 2 - ST5 y ST6}

\section{Entradas y salidas de la Fase 2 completa}
Como preparación para abordar la Fase 3 del modelo CaDQM, se presentan a continuación las entradas y salidas generales de la Fase 2.

\begin{center}
    \resizebox{\textwidth}{!}{ % Ajusta la tabla al ancho de la página
    \begin{tabular}{| p{8cm} | p{8cm} |} % Columnas con ancho ajustado
        \hline
        \multicolumn{2}{|c|}{\cellcolor{blue!30} \textbf{Entradas y salidas}} \\
        \hline
        \rowcolor{blue!15}
        \textbf{Entradas} & \textbf{Salidas} \\
        \hline
        Reporte del análisis de requerimientos de usuarios (\ref{ST3_reqUsuarios})& Reporte de problemas de CD priorizados (\ref{ST4_problemas_priorizados})\\
        \hline
        Reporte del análisis de datos (\ref{ST2_SubsectionAnalisis}) & Modelo de CD (\ref{ST4_modelo_calidad_contextual})\\
        \hline
        Reporte de problemas de CD (\autoref{Problemas_ST2}) &  Especificación de la BD de metadatos de CD (\ref{Diseño_BDQ})\\
        \hline
        Modelo de Contexto (\ref{ST3_SubsectionContexto}) & Reporte de medición de la CD (\ref{ImplementacionMetodos})\\
        \hline
         & Modelo de contexto (\ref{Contexto_ST5})\\
        \hline
         & Reporte de evaluación de CD (\ref{ImplementacionMetodos})\\
        \hline
    \end{tabular}
    }
\end{center}

\noindent CD: Calidad de datos.\\
BD: Base de datos.





\section{Descripción del desarrollo del trabajo}

% forma de trabajo, herramientas usadas, desafíos enfrentados, etc.
% resultados obtenidos en cada una de las actividades

Siguiendo la metodología CaQDM en la fase 2 el primer paso fue repasar los problemas de calidad analizados en la fase 1 y poder identificar potenciales nuevos problemas. 

 Una vez identificados, procedimos a asignar una prioridad a cada uno tomando en cuenta los requerimientos de calidad, las reglas de negocio, los requerimientos del sistema y las necesidades de filtrado del cliente.

Luego definimos la estructura de la base de datos que usaríamos para almacenar las distintas dimensiones, factores, métricas, métodos, métodos aplicados y ejecuciones de las distintas metricas. 

Una vez pronta la estructura de la base, procedimos a insertar todos los datos y a crear distintos procedimientos almacenados donde se encuentra la lógica de distintos algoritmos de los métodos aplicados. 
Con la estructura pronta y los datos cargados solo restaba ejecutar los distintos métodos aplicados y almacenar los resultados.

 Finalmente armamos el modelo de calidad con sus distintas métricas y métodos tomando en cuenta las prioridades de los problemas de calidad.

\section{Conclusiones de la fase 2}

 La implementación de la segunda fase del modelo CaDQM en este proyecto permitió establecer un enfoque estructurado para diagnosticar el estado actual de la calidad de los datos, considerando los problemas detectados y la prioridad asignada a cada uno.

Mediante las distintas etapas logramos cuantificar la calidad de los datos mediante métricas especificas obtenidas analizando la realidad del problema. Luego aplicando dichas métricas y manejando distintos umbrales pudimos transformar estas métricas en distintas valoraciones de calidad importantes para ayudarnos a entender el estado de calidad de los datos.

 Finalmente, logramos entender el origen de los posibles problemas de calidad que pueden presentar un conjunto de datos y, tomando en cuenta las distintas métricas métodos implementados junto a los umbrales que elegimos, pudimos obtener procedimientos que nos ayudarán a entender estado de la calidad de los datos estudiados de una forma fácil de entender y comunicar.